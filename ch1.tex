%   MSc Business Analytics Dissertation
%   Format based on skeleton template provided as part of module MIS40750
%
%   Title:     Optimising the design of buffer preparation in bioprocessing
%              facilities
%   Author:    Sean Tully
%
%   Chapter 1: Introduction
%
%   Change Control:
%   When     Who   Ver  What
%   -------  ----  ---  --------------------------------------------------------
%   06Jun16  ST    0.1  Begun 
%

\chapter{Introduction}\label{C.intro}

\begin{quote}
Well I don't think we're \emph{for} anything. We're just products of evolution.
You can say, ``Gee, your life must be pretty bleak if you don't think there's a
purpose.'' But I'm anticipating having a good lunch.

\hspace{2cm}--- James Watson, \emph{in conversation with Richard Dawkins}
\end{quote}


\section{Background}\label{S.intro1}

This chapter gives a brief outline of bioprocessing and explains the background
to the research.

\subsection{Bioprocessing}\label{SS.bioproc}

Before the advent of biotechnology, most therapeutics (medicines) were what are
now termed \emph{small molecule} drugs. The pharmaceutical industry was
concerned with the synthesis of these products via predominantly chemical
processes, such as reaction, distillation and  crystallisation.  Small molecule
drugs typically consist of tens or hundreds of atoms, such as paracetamol, which
has a molar mass of approximately 151 g/mol, or aspirin (acetylsalicylic acid),
which has a molar mass of approximately 180 g/mol.

With the advent of recombinant D.N.A technology, biochemists gained
the ability to program the D.N.A. of simple biological microorganisms such as
\textit{Escherichia coli} and, eventually, mammalian cells, such as those of the
Chinese Hamster (\textit{Cricetulus griseus}).  The genetic structure of these
cells could be modified to produce complex molecules, which had previously
proved difficult or impossible to synthesise by any other means.  The
biopharmaceutical industry is concerned with the synthesis of therapeutics via
such biological pathways.  These products are known by various names, such as
\emph{biopharmaceuticals}, \emph{protein therapeutics} or, colloquially, as
\emph{biotech drugs}.

The first protein therapeutic to be synthesised on a large scale using 
biological pathways was insulin, which is a hormone used to regulate metabolism
and is administered to individuals suffering from diabetes.  Human insulin has a
molar mass of approximately 5808 g/mol.  It was not possible to commercialy
synthesise insulin chemically and it was initially produced by extracting the
hormone from the pancreases of mammals such as cows or pigs.  In 1978, 
scientists working at  the American company Genentech successfully modified 
cells of \textit{E. coli} to produce insulin and this synthetic insulin was
first brought to market in 1982.

The industry that has grown up around the production of biopharmaceuticals is 
known as the \emph{biopharmaceutical industry}, or, colloquially, as the
\emph{biotech} or \emph{biopharma} industry.  McKinsey (http://www.mckinsey.com/industries/pharmaceuticals-and-medical-products/our-insights/rapid-growth-in-biopharma) ***NOTE: add as ref*** estimated that
the biopharmaceutical market had global revenues of \$163 billion per annum and 
was worth 20\% of the overall pharmaceutical market.  The McKinsey report also
notes that large-scale biopharmaceutical manufactring facilities typically cost
in the region of ``\$200 million to \$500 million or more'' to build.

\subsection{Bioprocess Engineering}\label{SS.bioproceng}

\begin{defn}
``Bioprocess engineering: A specialist branch of (chemical) engineering that
involves the design and operation of processes used for the production of
biological products such as foods, pharmaceuticals, and biopolymers.''
--- \citet{Schaschke:2014}.
\end{defn}

\subsection{Buffers and Media}\label{SS.buffmed}

\subsection{Proposal}\label{SS.proposal}
