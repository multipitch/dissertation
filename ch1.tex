%   MSc Business Analytics Dissertation
%   Format based on skeleton template provided as part of module MIS40750
%
%   Title:     Optimising the design of buffer preparation in bioprocessing
%              facilities
%   Author:    Sean Tully
%
%   Chapter 1: Introduction
%
%   Change Control:
%   When     Who   Ver  What
%   -------  ----  ---  --------------------------------------------------------
%   06Jun16  ST    0.1  Begun 
%

\chapter{Introduction}\label{C.intro}

\begin{quote}
We have some freedom in setting up our personal standards of beauty, but it is
especially nice when the things we regard as beautiful are also regarded by
other people as useful.

\hspace{2cm}--- Donald Knuth, \emph{Computer Programming as an Art}
\end{quote}


\section{Background}\label{S.intro1}

We begin by \ldots

\subsection{Particular Background}\label{SS.xyz}

Here are some examples of indexing: Newton's algorithm\index{Newton's algorithm}\index{algorithm!Newton} is 
still widely used, with modifications.  

Note that the \verb|\index{algorithm!Newton}| gives an index subentry for Newton under the entry for algorithm.
The index entries and/or their page numbers can be formatted using a pipe $|$ symbol in the \verb!\index{}! 
command as follows:

\begin{defn}
The strictest definition of an \emph{algorithm}\index{algorithm@\textit{algorithm}} is: a finite set of instructions 
that can be carried out in a finite amount of time: that is, it must terminate.

These instructions must be clear and unambiguous as they are to be interpreted by a (dumb) 
machine, so we must be absolutely precise about their meaning --- mathematical logic is 
thus crucial in the design of algorithms\index{algorithm@\textbf{algorithm}}.
\end{defn}

In practice, many useful numerical ``algorithms''\index{algorithm|textbf} that we study may get closer and closer 
to the desired solution without reaching it in a finite time.  So, typically, we accept as an 
``algorithm''\index{algorithm|textit} a finite set of instructions that will get within any desired tolerance 
of the true solution in a finite time.
If the algorithm is stochastic (involves probability, as many modern ones do) the term 
``metaheuristic''\index{metaheuristic|textbf} is sometimes used.

In particular, you could use the \verb|\index{algorithm@\textit{algorithm}}| or \verb!\index{algorithm|\textbf}! to
indicate the first or most important occurrence in the text of the term ``algorithm'', etc.

Some minor examples of other things indexing can do:
\begin{itemize}
\item You can handle accented words as in \'ecole\index{ecole@\'ecole}: the index entry appears in the correct 
order under E, as desired;
\item You can put in cross-references, as in 

Are metaheuristics\index{metaheuristic|see{algorithm}} really algorithms\index{algorithm|seealso{metaheuristic}}?
\end{itemize}

Note: when you LaTeX your file \texttt{myfile.tex}, a file \texttt{myfile.idx} is produced by \verb|\makeindex|; 
this file must be sorted by an operating system command, e.g.,

\texttt{makeindex myfile} 

This generated a \emph{sorted} index file \texttt{myfile.ind}.  Running LaTeX one more time gets the index printed 
in the right place by \verb|\printindex|.

Here is a dummy theorem to show how to reference notation:
\begin{thm}\label{Th.FF.fte.field}
Let $\FF_q$ be a finite field of $q$ elements.  Then $q$ is a power of some prime number $p$.
\end{thm}

