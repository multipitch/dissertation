\documentclass{beamer}
\usetheme{Boadilla}
\usepackage{siunitx}
\usepackage{amsmath}
\usepackage{graphicx}

\title[Bioprocessing facilities design]{Dissertation:
    Optimising the design of buffer preparation in bioprocessing facilities}
\subtitle{MSc in Business Analytics Part Time 2015--2017}
\author{Sean Tully}
\institute{University College Dublin}
\date{25\textsuperscript{th} August 2017}

\begin{document}
\setbeamertemplate{caption}{\raggedright\insertcaption\par}

\begin{frame}
    \titlepage
\end{frame}


\begin{frame}
    \frametitle{Introduction -- Problem Definition}
    \begin{columns}
        \column{0.65\textwidth}
            \begin{itemize}
                \item Bioprocessing involves the synthesis of therapeutic
                    products using biological agents (bacteria or mammalian
                    cells)        
                \item The purification of these products typically requires the
                    use of 10--20 different buffer solutions (\emph{buffers})
                \item Calculating the correct number and volume of vessels for
                    buffer preparation in a large-scale facility represents a
                    significant engineering challenge
                \item The aim of this work was to develop a methodology for
                    calculating buffer vessel requirements, given some process
                    information
            \end{itemize}        
        \column{0.35\textwidth}
        \begin{figure}
            \centering
            \includegraphics[angle=0,scale=0.22]{alexion.jpg}\\
            \vspace{0.3cm}
            \includegraphics[angle=0,scale=0.22]{centocor.jpg}
            \caption{Alexion, Dublin and Centocor, Cork \copyright PM Group}
        \end{figure}
    \end{columns}
\end{frame}

\begin{frame}
    \frametitle{Buffer Preparation}
    \begin{columns}
        \column{0.5\textwidth}
        \begin{itemize}
            \item Buffer preparation requires two vessels
            \item Each buffer is prepared in a \emph{preparation vessel}, which
                may be used to prepare many buffers
            \item It is assumed that each buffer has a dedicated \emph{hold
                vessel}
            \item We want to minimise the total cost of vessels in buffer
                preparation.
        \end{itemize}
        \column{0.5\textwidth}
        \begin{figure}
            \centering
            \includegraphics[angle=0,scale=0.5]{./../figures/buffer_pfd.pdf}
            \caption{Single buffer preparation -- equipment}
        \end{figure}
    \end{columns}
\end{frame}

\begin{frame}
    \frametitle{Buffer Preparation Scheduling}
    \begin{figure}
        \centering
        \includegraphics[angle=0,scale=0.45]{./../figures/explanatory.pdf}
        \caption{Single buffer preparation -- scheduling}
    \end{figure}
    \begin{itemize}
    \item The volume of each buffer is known, as is it's time of first use and
        total duration of use
    \item Some flexibility exists in the hold duration and in vessel selection
    \end{itemize}
\end{frame}

\begin{frame}
    \frametitle{Global parameters for random example}
    \begin{table}
        \begin{tabular}{l | l | r | c}
            symbol & short description & value & unit\\ \hline
            $T$ & process cycle time & 96.0 & h\\
            $\Delta t_{\mathit{PREP,PRE}}$ & prep pre duration & 12.0 & h\\
            $\Delta t_{\mathit{PREP,POST}}$ & prep post duration & 1.5 & h\\
            $\Delta t_{\mathit{TRANSFER}}$ & transfer duration & 2.0 & h\\
            $\Delta t_{\mathit{HOLD,PRE}}$ & hold pre duration & 8.0 & h\\
            $\Delta t_{\mathit{HOLD,POST}}$ & hold post duration & 1.5 & h\\
            $\Delta t_{\mathit{HOLD,MIN}}$ & minimum hold duration & 12.0 & h\\
            $\Delta t_{\mathit{HOLD,MAX}}$ & maximum hold duration & 60.0 & h\\
            $f_{\mathit{MINFILL}}$ & minimum fill ratio & 0.3 & --\\
            $f_{\mathit{UTIL}}$ & maximum utilisation ratio & 0.8 & --\\
        \end{tabular}
    \end{table}
\end{frame}

\begin{frame}
    \frametitle{Vessel data for random example}
    \begin{table}
        \begin{tabular}{l | r | r}
            names & volumes & costs\\
            & $V_{m}$ (l) & $c_{m}$ (--)\\\hline
            \SI{1000}{\litre} & \SI{1000.0}{} & \SI{63.10}{}\\
            \SI{2000}{\litre} & \SI{2000.0}{} & \SI{95.64}{}\\
            \SI{3000}{\litre} & \SI{3000.0}{} & \SI{121.98}{}\\
            \SI{4000}{\litre} & \SI{4000.0}{} & \SI{144.96}{}\\
            %\SI{5000}{\litre} & \SI{5000.0}{} & \SI{165.72}{}\\
            %\SI{6000}{\litre} & \SI{6000.0}{} & \SI{184.88}{}\\
            %\SI{8000}{\litre} & \SI{8000.0}{} & \SI{219.71}{}\\
            %\SI{10000}{\litre} & \SI{10000.0}{} & \SI{251.19}{}\\
            %\SI{12000}{\litre} & \SI{12000.0}{} & \SI{280.83}{}\\
            %\SI{16000}{\litre} & \SI{16000.0}{} & \SI{333.02}{}\\
            %\SI{18000}{\litre} & \SI{18000.0}{} & \SI{357.41}{}\\
            $\quad\vdots$ & $\vdots\quad$ & $\vdots\quad$\\
            \SI{20000}{\litre} & \SI{20000.0}{} & \SI{380.73}{}\\
            \SI{22000}{\litre} & \SI{22000.0}{} & \SI{403.14}{}\\
            \SI{25000}{\litre} & \SI{25000.0}{} & \SI{435.28}{}\\
            \SI{30000}{\litre} & \SI{30000.0}{} & \SI{485.59}{}\\
        \end{tabular}
    \end{table}
\end{frame}

\begin{frame}
    \frametitle{Buffer data for random example}
    \begin{table}
        \centering
        \label{tbl.buffer}
        \begin{tabular}{l | c | c | c}
            names & required volumes & use start times & use durations\\
            & $U_{n}$ (l) & $t_{\mathit{USE},n}^{*}$ (h) 
            & $\Delta t_{\mathit{USE},n}$
            (h)\\ \hline
            \text{Buffer \#1} & \SI{24427.13}{} & \SI{76.23}{} & \SI{20.56}{}\\
            \text{Buffer \#2} & \SI{5487.29}{} & \SI{0.21}{} & \SI{49.77}{}\\
            \text{Buffer \#3} & \SI{2588.36}{} & \SI{25.78}{} & \SI{24.56}{}\\
            \text{Buffer \#4} & \SI{7102.05}{} & \SI{46.79}{} & \SI{27.77}{}\\
            \text{Buffer \#5} & \SI{1020.87}{} & \SI{87.7}{} & \SI{36.58}{}\\
            \text{Buffer \#6} & \SI{19508.79}{} & \SI{35.52}{} & \SI{58.53}{}\\
            \text{Buffer \#7} & \SI{23073.55}{} & \SI{42.26}{} & \SI{39.71}{}\\
            \text{Buffer \#8} & \SI{25454.10}{} & \SI{48.38}{} & \SI{43.47}{}\\
            \text{Buffer \#9} & \SI{24088.67}{} & \SI{4.18}{} & \SI{55.41}{}\\
            \text{Buffer \#10} & \SI{3172.46}{} & \SI{48.31}{} & \SI{23.27}{}\\
            \text{Buffer \#11} & \SI{24752.71}{} & \SI{76.38}{} & \SI{45.80}{}\\
            \text{Buffer \#12} & \SI{13445.31}{} & \SI{73.93}{} & \SI{34.25}{}\\
        \end{tabular}
    \end{table}
\end{frame}

\begin{frame}
    \frametitle{Basic Model}
    Objective Function: Minimise total preparation vessel cost\\
    \begin{equation}
        \boldsymbol{Z} = \sum_{m \in \mathcal{M}} \sum_{p \in \mathcal{P}} c_m
        \boldsymbol{y}_{mp}
    \end{equation}
    Subject to \ldots
    \begin{itemize}
    \item Each buffer must be prepared in exactly one slot\\
    \begin{equation}
        \sum_{p \in \mathcal{P}} \boldsymbol{x}_{np} = 1 \quad \forall n \in 
        \mathcal{N}
    \end{equation}
    \item Each slot may contain at most one vessel instance
    \begin{equation}
        \sum_{m \in \mathcal{M}} \boldsymbol{y}_{mp} \le 1 \quad \forall p \in 
        \mathcal{P}
    \end{equation}
    \end{itemize}
\end{frame}

\begin{frame}
    \frametitle{Basic Model (continued)}
    \begin{itemize}
    \item Each vessel must be adequately sized for all buffers prepared in it
    \begin{equation}
        U_{n} \boldsymbol{x}_{np} \le \sum_{m \in \mathcal{M}} V_{m}
        \boldsymbol{y}_{mp} \quad \forall n \in \mathcal{N}, \enspace \forall
        p \in \mathcal{P}
    \end{equation}
    \begin{equation}
        V_{\mathit{MAX}} + U_{n} \ge 
        f_{\mathit{MINFILL}} \sum_{m \in \mathcal{M}} V_{m} \boldsymbol{y}_{mp}
        + V_{\mathit{MAX}} \boldsymbol{x}_{np} 
        \quad \forall n \in \mathcal{N}, \enspace \forall p \in \mathcal{P}
    \end{equation}
    \item Each preparation slot must not be too busy
    \begin{equation}
        \Delta t_{\mathit{PREP}} \sum_{n \in \mathcal{N}} \boldsymbol{x}_{np}
        \le f_{\mathit{UTIL}} T \quad \forall p \in \mathcal{P}
    \end{equation}
    \end{itemize}
\end{frame}

\begin{frame}
    \frametitle{Complete Model}
    \begin{itemize}
        \item The complete model adds scheduling constraints to the basic
        model
        \item The general scheduling constraint can be stated quite simply:\\
        \emph{`Two events requiring the same piece of equipment must not occur
        simultaneously'}
        \item To state the above constraint mathematically is not so simple
        \item The additional constraints required will now be described
        \ldots
    \end{itemize}
\end{frame}

\begin{frame}
    \frametitle{Complete Model -- Additional constraints}
    \begin{itemize}
        \item Hold procedures must not clash
        \begin{multline}
            \Delta t_{\mathit{HOLD,PRE}} + \Delta t_{\mathit{TRANSFER}}
            + \boldsymbol{z}_{n} + \Delta t_{\mathit{USE},n} 
            + \Delta t_{\mathit{HOLD,POST}} \le T\\ \quad \forall n \in 
            \mathcal{N}
        \end{multline}
        \item Identify if a pair of distinct buffers are both prepared in a
        \emph{particular} slot (note: cubic)
        \begin{equation}
            \begin{split}
                \begin{alignedat}{11}
                    &\boldsymbol{w}_{nkp} {}&&\le{} \tfrac{1}{2} 
                    &&\boldsymbol{x}_{np}
                    {}+{} &&\tfrac{1}{2} && \boldsymbol{x}_{kp} &{}-{} 1\\
                    &\boldsymbol{w}_{nkp} {}&&\ge{} &&\boldsymbol{x}_{np} {}+{}
                    && && \boldsymbol{x}_{kp} &{}-{} 1\\
                \end{alignedat}
            \end{split}
            \quad
            \begin{split}
                \forall n \in \mathcal{N}, \enspace \forall k \in \mathcal{N};
                \; k > n, \enspace \forall p \in \mathcal{P}
            \end{split}
            \label{eq.w1}
        \end{equation}
        \item The above can be used to tell us if  a pair of distinct buffers
        are both prepared in the \emph{same} slot
        \begin{equation}
            \boldsymbol{a}_{nk} = \sum_{p \in \mathcal{P}} \boldsymbol{w}_{nkp}
            \quad \forall n \in \mathcal{N}, \enspace \forall k \in 
            \mathcal{N}; \; k > n
            \label{eq.a1}
        \end{equation}
    \end{itemize}
\end{frame}

\begin{frame}
    \frametitle{Complete Model -- Additional constraints (continued)}
    \begin{itemize}
        \item We now define a number of additional constraints which are
        required due to the cyclic nature of the scheduling problem
        \begin{equation}
            \begin{split}
                \begin{alignedat}{2}
                    T \boldsymbol{q}_{n} - \boldsymbol{z}_{n} &\ge
                    - t_{\mathit{USE},n}\\
                    T \boldsymbol{q}_{n} - \boldsymbol{z}_{n} &\le
                    - t_{\mathit{USE},n} + T\\
                \end{alignedat}
            \end{split}
            \quad \forall n \in \mathcal{N}
        \end{equation}
        \begin{equation}
            \begin{split}
                \begin{alignedat}{2}
                    - T \boldsymbol{q}_{n} + T \boldsymbol{r}_{n} 
                    + \boldsymbol{z}_{n}
                    &\le t_{\mathit{USE},n} + \Delta t_{\mathit{PREP}}\\
                    - T \boldsymbol{q}_{n} + T \boldsymbol{r}_{n}
                    + \boldsymbol{z}_{n}
                    &\ge t_{\mathit{USE},n} + \Delta t_{\mathit{PREP}} - T\\
                    \end{alignedat}
                \quad \forall n \in \mathcal{N}
            \end{split}
        \end{equation}
        \begin{equation}
            \begin{split}
                \begin{alignedat}{2}
                    T \boldsymbol{q}_{n} + T \boldsymbol{s}_{n} 
                    - \boldsymbol{z}_{n}
                    &\le -t_{\mathit{USE},n} + \Delta t_{\mathit{PREP}}\\
                    T \boldsymbol{q}_{n} + T \boldsymbol{s}_{n} 
                    - \boldsymbol{z}_{n}
                    &\ge -t_{\mathit{USE},n} + \Delta t_{\mathit{PREP}} + T\\
                    \end{alignedat}
                \quad \forall n \in \mathcal{N}
            \end{split}
        \end{equation}
        \begin{equation}
            \begin{split}
                \begin{alignedat}{8}
                    &&\boldsymbol{r}_{n} && {}+{} &&\boldsymbol{s}_{n} && {}-{} 
                    &&\boldsymbol{u}_{n} &\ge 0\\
                    &&\boldsymbol{r}_{n} && {}+{} &&\boldsymbol{s}_{n} && {}-{} 
                    &2&\boldsymbol{u}_{n} &\le 0\\
                \end{alignedat}
                \quad \forall n \in \mathcal{N}
            \end{split}
        \end{equation}    
    \end{itemize}
\end{frame}

\begin{frame}
    \frametitle{Complete Model -- Additional constraints (continued)}
    \begin{itemize}
    \item The preparation scheduling constraint can now be defined
    \begin{equation}
            \begin{split}
                \begin{aligned}
                    T \boldsymbol{q}_{k} - T \boldsymbol{q}_{n}
                    + 2T \boldsymbol{u}_{n} + 2T \boldsymbol{v}_{nk}
                    - 2T \sum_{p \in \mathcal{P}} \boldsymbol{w}_{nkp} 
                    - \boldsymbol{z}_{k} + \boldsymbol{z}_{n}\\
                    \ge t_{\mathit{USE},n} - t_{\mathit{USE},k}
                    + \Delta t_{\mathit{PREP}} - 2T
                \end{aligned}\\
                \begin{aligned}
                    T \boldsymbol{q}_{k} - T \boldsymbol{q}_{n}
                    - 2T \boldsymbol{u}_{n} + 2T \boldsymbol{v}_{nk}
                    + 2T \sum_{p \in \mathcal{P}} \boldsymbol{w}_{nkp} 
                    - \boldsymbol{z}_{k} + \boldsymbol{z}_{n}\\
                    \ge t_{\mathit{USE},n} - t_{\mathit{USE},k}
                    - \Delta t_{\mathit{PREP}} + 4T
                \end{aligned}\\
                \begin{aligned}
                    T \boldsymbol{q}_{k} - T \boldsymbol{q}_{n}
                    - T \boldsymbol{r}_{n} + 2T \boldsymbol{u}_{n}
                    + 2T \sum_{p \in \mathcal{P}} \boldsymbol{w}_{nkp} 
                    - \boldsymbol{z}_{k} + \boldsymbol{z}_{n}\\
                    \ge t_{\mathit{USE},n} - t_{\mathit{USE},k}
                    - \Delta t_{\mathit{PREP}} + 4T
                \end{aligned}\\
                \begin{aligned}
                    T \boldsymbol{q}_{k} - T \boldsymbol{q}_{n}
                    + T \boldsymbol{s}_{n} - 2T \boldsymbol{u}_{n}
                    - 2T \sum_{p \in \mathcal{P}} \boldsymbol{w}_{nkp} 
                    - \boldsymbol{z}_{k} + \boldsymbol{z}_{n}\\
                    \ge t_{\mathit{USE},n} - t_{\mathit{USE},k}
                    + \Delta t_{\mathit{PREP}} - 4T
                \end{aligned}\\
                \begin{aligned}
                    \forall n \in \mathcal{N}, \enspace \forall k 
                    \in \mathcal{N}; \; k > n
                \end{aligned}\\
            \end{split}
        \end{equation}
    \end{itemize}
\end{frame}

\begin{frame}
    \frametitle{Complete Model -- Variables}
    \begin{equation}
        \boldsymbol{q}_{n} \in \left\{0, 1\right\} \quad \forall n \in 
        \mathcal{N}
    \end{equation}
    \begin{equation}
        \boldsymbol{r}_{n} \in \left\{ 0, 1 \right\} \quad \forall n \in 
        \mathcal{N}
    \end{equation}
    \begin{equation}
        \boldsymbol{s}_{n} \in \left\{ 0, 1 \right\} \quad \forall n \in 
        \mathcal{N}
    \end{equation} 
    \begin{equation}
        \boldsymbol{u}_{n} \in \left\{ 0, 1 \right\} \quad \forall n \in 
        \mathcal{N}
    \end{equation}
    \begin{equation}
        \boldsymbol{v}_{nk} \in \left\{ 0, 1 \right\} \quad \forall n \in 
        \mathcal{N}, \enspace \forall k \in \mathcal{N}; \; k > n
    \end{equation}
    \begin{equation}
        \boldsymbol{w}_{nkp} \in \left\{ 0, 1 \right\} \quad \forall n \in 
        \mathcal{N}, \enspace \forall k \in \mathcal{N}; \; k > n, \enspace
        \forall p \in \mathcal{P}
    \end{equation}
    \begin{equation}
        \boldsymbol{x}_{np} \in \left\{ 0, 1 \right\} \quad \forall n \in 
        \mathcal{N}, \enspace \forall p \in \mathcal{P}
    \end{equation}
    \begin{equation}
        \boldsymbol{y}_{mp} \in \left\{ 0, 1 \right\} \quad \forall m \in 
        \mathcal{M}, \enspace \forall p \in \mathcal{P}
    \end{equation}
    \begin{equation}
        \Delta t_{\mathit{HOLD,MIN}} \le \boldsymbol{z}_{n} \le 
        \Delta t_{\mathit{HOLD,MAX}}; \quad
        \boldsymbol{z}_{n} \in \mathbb{R} \quad \forall n \in \mathcal{N}
    \end{equation}
\end{frame}

\begin{frame}
    \frametitle{Binary Variables in the Scheduling Constraint}
    The binary var
    \begin{table}
        \centering
        \small
        \begin{tabular}{c c c c c | c}
            $\boldsymbol{a}_{nk}$ & $\boldsymbol{u}_{n}$ & $\boldsymbol{v}_{n}$
            & $\boldsymbol{r}_{n}$ & $\boldsymbol{s}_{n}$
            & $\boldsymbol{t}_{\mathit{PREP},k} = \bullet$\\\hline
            0 & $\ast$ & $\ast$ & $\ast$ & $\ast$ 
            & (no active scheduling constraints)\\
            1 & 0 & 0 & $\ast$ & $\ast$
            & $\boldsymbol{t}_{\mathit{PREP},n} + \Delta t_{\mathit{PREP}} \le
                \bullet \le T$\\
            1 & 0 & 1 & $\ast$ & $\ast$ 
            & $0 \le \bullet \le \boldsymbol{t}_{\mathit{PREP},n} 
                - \Delta t_{\mathit{PREP}}$\\
            1 & 1 & $\ast$ & 0 & 0 
            & $\boldsymbol{t}_{\mathit{PREP},n} + \Delta t_{\mathit{PREP}}
                \le \bullet \le \boldsymbol{t}_{\mathit{PREP},n} 
                - \Delta t_{\mathit{PREP}}$\\
            1 & 1 & $\ast$ & 0 & 1 
            & $\boldsymbol{t}_{\mathit{PREP},n} + \Delta t_{\mathit{PREP}} - T
                \le \bullet \le \boldsymbol{t}_{\mathit{PREP},n}
                - \Delta t_{\mathit{PREP}}$\\
            1 & 1 & $\ast$ & 1 & 0 
            & $\boldsymbol{t}_{\mathit{PREP},n} + \Delta t_{\mathit{PREP}}
                \le \bullet \le \boldsymbol{t}_{\mathit{PREP},n}
                - \Delta t_{\mathit{PREP}} + T$\\
            1 & 1 & $\ast$ & 1 & 1 
            & $\boldsymbol{t}_{\mathit{PREP},n} + \Delta t_{\mathit{PREP}} - T
                \le \bullet \le \boldsymbol{t}_{\mathit{PREP},n}
                - \Delta t_{\mathit{PREP}} + T$\\
        \end{tabular}
    \end{table}
\end{frame}

\begin{frame}
    \frametitle{Implementation}
    \begin{itemize}
        \item The Complete Model was implemented in python, using the PuLP
            library which acts as an API to several commercial and open-source
            mixed-integer linear programming solvers
        \item The proprietary IBM ILOG CPLEX Optimizer was primarily used to
            solve the problem
        \item The open-source CBC solver was also used, though it is
            considerably slower
        \item The matplotlib library was used to generate plots
        \item Both random data and data from real-world examples were examined
    \end{itemize}
\end{frame}

%\begin{frame}
%    \frametitle{Complexity}
%    For a problem with $M$ vessels and $N$ buffers, there
%    are:
%    \begin{itemize}
%        \item $N {{N}\choose{2}} + N^2 
%        + {{N}\choose{2}} + NM + 5N$
%        variables
%        \item $2N{{N}\choose{2}} 
%        + 4{{N}\choose{2}} + 2N^2 + 12N + 1$
%        equations
%    \end{itemize}
%    \vspace{0.5cm}
%    For the random example detailed in this presentation, there are:
%    \begin{itemize}
%        \item 1242 variables
%        \item 2281 equations
%    \end{itemize}
%\end{frame}

\begin{frame}
    \frametitle{Complexity Plots}
    \begin{columns}
        \column{0.5\textwidth}
        \begin{figure}
            \centering
            \includegraphics[angle=0,scale=0.55]{equations.pdf}
        \end{figure}
        \column{0.5\textwidth}
        \begin{figure}
            \centering
            \includegraphics[angle=0,scale=0.55]{variables.pdf}
        \end{figure}
    \end{columns}
\end{frame}

\begin{frame}
    \frametitle{Results -- Random Example}
    Solving the random example yields the following result:
        \begin{table}
            \centering
            \caption{Required preparation vessels for random example}
            \begin{tabular}{r}
                vessel size\\ \hline
                \SI{2000}{\litre}\\
                \SI{8000}{\litre}\\
                \SI{25000}{\litre}\\
                \SI{30000}{\litre}\\
            \end{tabular}
        \end{table}
    Note that there may be many ways of assigning the buffers to these vessels
    and many feasible ways of scheduling these buffers.
    
    A feasible schedule may be represented graphically using an \emph{equipment
    time utilisation} plot.
\end{frame}

\begin{frame}
    \frametitle{Equipment Time Utilisation -- Random Example}
    \begin{figure}
        \centering
        \includegraphics[angle=0,scale=0.55]{./../figures/plot1.pdf}
    \end{figure}
\end{frame}

\begin{frame}
    \frametitle{Goal Programming -- Reducing Hold Times}
    A \emph{better} feasible schedule may be obtained using \emph{goal
    programming}.
    
    The (original) objective function value is fixed as a parameter and we
    replace the original objective function with a new one, which seeks to 
    minimise hold times.
    
    In doing so, hold times are minimised, but only to the extent that they
    don't interfere with the original list of required preparation vessels.
    \begin{figure}
        \centering
        \includegraphics[width=0.495\textwidth]{./../figures/plot1.pdf}
        %\hfill
        \includegraphics[width=0.495\textwidth]{./../figures/plot2.pdf}
    \end{figure}
\end{frame}

\begin{frame}
    \frametitle{Solution Time}
    \begin{figure}
        \centering
        \includegraphics[angle=0,scale=0.75]{timing.pdf}
    \end{figure}
\end{frame}

\begin{frame}
    \begin{columns}
        \column{0.15\textwidth}
            \begin{figure}
                \centering
                \includegraphics[angle=0,scale=0.03]{ucd_logo.png}
            \end{figure}
        \column{0.7\textwidth}
        \centering MSc in Business Analytics -- Dissertation\\
        \large \textbf{Optimising the design of buffer 
                       preparation in bioprocessing facilities}
        \column{0.15\textwidth}
            \begin{figure}
                \centering
                \includegraphics[angle=0,scale=0.03]{ucd_logo.png}
            \end{figure}
    \end{columns}
    \centering \small \emph{Author: Sean Tully \quad Supervisor: Prof. Michael
        O'Neill}\par
    \vspace{0.5cm}
    \begin{columns}
        \column{0.4\textwidth}
            %This is some text that i have written. This is some text that i have written. This is some text that i have written. This is some text that i have written. This is some text that i have written. This is some text that i have written. This is some text that i have written. This is some text that i have written. This is some text that i have written. This is some text that i have written. This is some text that i have written. This is some text that i have written.
            \scriptsize \textbf{Aim}:
            \begin{itemize}
                \item  Development of a methodology and tool
                for finding the optimum number, size and assignment of buffer
                preparation vessels in the design of a large-scale bioprocess
                facility
            \end{itemize}
            \textbf{Key Achievements}:
            \begin{itemize}
                \item The problem was modelled as a MILP problem.
                \item The MILP problem was implemented in code.
                \item Generation of graphical production schedules 
            \end{itemize}

        \column{0.6\textwidth}
            \begin{figure}
                \centering
                \includegraphics[width=0.95\textwidth]{./../figures/plot2.pdf}
            \end{figure}  
    \end{columns}
\end{frame}

\end{document}
