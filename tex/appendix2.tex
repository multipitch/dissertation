%   MSc Business Analytics Dissertation
%   Format based on skeleton template provided as part of module MIS40750
%
%   Title:     Optimising the design of buffer preparation in bioprocessing
%              facilities
%   Author:    Sean Tully
%
%   Appendix 2: program code
%
%   Change Control:
%   When     Who   Ver  What
%   -------  ----  ---  --------------------------------------------------------
%   06Jun16  ST    0.1  Begun 
%

\chapter{Program code description}\label{C.Appendix2}

{\Large Classes in \texttt{model.py}}

The \texttt{model.py} file contains the following classes; a brief overview of
each class is given in the paragraphs hereafter.
\begin{itemize}
    \item \texttt{Parameters}
    \item \texttt{Vessels}
    \item \texttt{Buffers}
    \item \texttt{Variable}
    \item \texttt{Variables}
    \item \texttt{Results}
\end{itemize}

%\paragraph{Parameters}
The \texttt{Parameters} class is used to store parameters data (see 
\hyperref[S.parameters]{Section \ref*{S.parameters}}). 
It can read the data from a file, or be passed the data as a \texttt{dict}.

%\paragraph{Vessels}
The \texttt{Vessels} class is used to store vessel data (see 
\hyperref[S.vesseldata]{Section \ref*{S.vesseldata}}).
Similar to the \texttt{Parameters} class, it can read the data from a file, or
be passed the data as a \texttt{dict}.

%\paragraph{Buffers}
The \texttt{Buffers} class is used to store buffer data (see 
\hyperref[S.bufferdata]{Section \ref*{S.bufferdata}}). 
Again, it can read the data from a file, or be passed the data as a
\texttt{dict}. It is also used to store some results when a problem has been
solved; it contains a class method, \texttt{get\_results} for this purpose.

%\pararaph{Variable}
The \texttt{Variable} class is used to define a variable, e.g.\
$\boldsymbol{z}_{n}$ or $\boldsymbol{w}_{nkp}$. 
From the point of view of the \texttt{pulp} module, a variable is a single
scalar entity (a \texttt{pulp.LpVariable} object).
An instance of the \texttt{Variable} class represents all the elements of a 
multidimensional variable e.g.\
$\boldsymbol{x}_{np} \enspace \forall n \in N, \enspace \forall p \in P$.
The class is initialised with information on the multidimensional variable,
such as dimensions, bounds and whether the variable is integer or real valued.
When the model is solved, the \texttt{evaluate} class method is used to create
a multidimensional array of the evaluated values of each scalar variable
object.

%\pararaph{Variables}
The \texttt{Variables} class simply exists to contain a group of
\texttt{Variable} class instances.

%\pararaph{Results}
The \texttt{Results} class is a container for holding the solution variables.
It has an \texttt{add} method for populating itself with data.

\vspace{1cm}
{\Large Key functions in \texttt{model.py}}

The \texttt{model.py} file also contains the following functions:
\begin{itemize}
    \item \texttt{csv\_columns\_to\_dict\_of\_lists}
    \item \texttt{variable\_iterator}
    \item \texttt{variable\_iterator\_loc}
    \item \texttt{\_variable\_iterator\_loc}
    \item \texttt{unflatten\_variable}
    \item \texttt{define\_problem}
    \item \texttt{initial\_objective}
    \item \texttt{secondary\_objective}
    \item \texttt{generate\_random\_model}
    \item \texttt{run\_primary}
    \item \texttt{run\_secondary}
    \item \texttt{standard\_run}
    \item \texttt{run\_random\_models}
    \item \texttt{many\_random}
    \item \texttt{one\_random}    
\end{itemize}

The \texttt{csv\_columns\_to\_dict\_of\_lists} function reads from csv files
where the header row contains keys and subsequent rows contain values.
It converts this data to a \texttt{dict}, with the header row entries as keys
and the columns of data below a header as a list of values for that key.
This is used to read from the \texttt{buffers.csv} and \texttt{vessels.csv}
files.

The \texttt{variable\_iterator} and \texttt{variable\_iterator\_loc}
functions are recursive functions that may be used to access the components of
a multidimensional variable in order.
The former just returns the component, whilst the latter also returns its
index.
Both functions are wrappers to the function \texttt{\_variable\_iterator\_loc},
which actually performs the recursion.
These are required to flatten a multidimensional variable into a list so that
it is suitable for passing to \texttt{pulp}.

The \texttt{unflatten\_variable} variable function essentially performs the
reverse of the above iterator functions; given a required set of dimensions
and a list of objects, it unflattens the list into a multidimensional array of
objects with the requisite dimensions.
It calls \texttt{variable\_iterator\_loc} to do this, i.e.\ it also operates
recursively.

The \texttt{define\_problem} function implements all of the equations in
\hyperref[SS.completesummary]{Section \ref*{SS.completesummary}}, except for
the objective function. It starts by initialising a problem 
instance (i.e.\ a \texttt{pulp.LpProblem} object), then defines all the
variables used in the equations, then adds all the equations to the problem
instance. The objective function is omitted because this allows the function to
be used for both the primary and the secondary objective runs.

The \texttt{initial\_objective} function sets the objective function as per
\hyperref[eq.objfn]{equation \ref*{eq.objfn}}.  This is used, after
\texttt{define\_problem} to fully define the complete problem summarised in 
\hyperref[SS.completesummary]{Section \ref*{SS.completesummary}}.

Similarly, the \texttt{secondary\_objective} function sets the objective
function as per \hyperref[eq.objfn2]{equation \ref*{eq.objfn2}} and also
implements the additional constraint required by the secondary problem
(\hyperref[eq.constr10]{equation \ref*{eq.constr10}}).
This function is used to define the secondary problem, as summarised in
complete problem summarised in 
\hyperref[SS.model2summary]{Section \ref*{SS.model2summary}}.

The remaining functions are used to generate models from defined or random
data and to run the models via pulp.

\vspace{1cm}
{\Large Key functions in \texttt{plot.py}}

The \texttt{plot.py} file contains a number of functions, listed below, for
generating various plots.  It uses the \texttt{matplotlib} library extensively.
\begin{itemize}
    \item \texttt{single\_cycle\_plot}
    \item \texttt{explanatory\_plot}
    \item \texttt{sched\_plot\_single}
    \item \texttt{sched\_plot\_all}
    \item \texttt{complexity\_plot}
    \item \texttt{timing\_plot}
    \item \texttt{cyclic\_xranges}
    \item \texttt{read\_durations}
\end{itemize}

The \texttt{single\_cycle\_plot} function produces an equipment time
utilisation plot for a solved problem. Examples in this document include
figures \ref{fig.primary} and \ref{fig.secondary} in this chapter.  This is the
main visual output from a solved problem.

\sloppy
The \texttt{explanatory\_plot} function produces the plot in
\hyperref[fig.explanatory]{Figure \ref*{fig.explanatory}}.
The \texttt{sched\_plot\_single} and \texttt{sched\_plot\_all} functions are
used to produce the plots in Figures \ref{fig.sched1} and \ref{fig.sched2}
respectively.
The \texttt{complexity\_plot} function produces is used to produce the plots
in Figures \ref{fig.dims} and \ref{fig.eqns}.
These are all static, explanatory plots used to illustrate the
report only.

\fussy
The \texttt{timing\_plot} function is used to generate a box-plot showing the 
duration taken to solve a random problem as a function of the number of buffers
in the problem. \hyperref[fig.timing]{Figure \ref*{fig.timing}} is an example
of such a plot.

The \texttt{cyclic\_xranges} function is called by the
\texttt{single\_cycle\_plot} function and the \texttt{sched\_plot\_all}
function.
If an operation overlaps the single cycle range boundary, it splits the
operation into two segments, one from the start of the operation to $T$ and the
other from $t=0$ to the end of the operation.

The \texttt{read\_durations} function is used to read timing data from a file.
This may be used by \texttt{timing\_plot}.  The capability to read and write
duration data to a file is useful as it can take a considerable amount of time
to obtain the duration data and if it was destroyed when the program
terminates, tweaking the appearance of the plot would otherwise require
re-running all the constituent simulations again.
