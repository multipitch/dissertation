%   MSc Business Analytics Dissertation
%   Format based on skeleton template provided as part of module MIS40750
%
%   Title:     Optimising the design of buffer preparation in bioprocessing
%              facilities
%   Author:    Sean Tully
%
%   Chapter 1: Introduction
%
%   Change Control:
%   When     Who   Ver  What
%   -------  ----  ---  --------------------------------------------------------
%   06Jun16  ST    0.1  Begun 
%

\chapter{Introduction}\label{C.intro}

\begin{quote}
Well I don't think we're \emph{for} anything. We're just products of evolution.
You can say, `Gee, your life must be pretty bleak if you don't think there's a
purpose.' But I'm anticipating having a good lunch.

\hspace{2cm}--- James Watson, \emph{in conversation with Richard Dawkins}
\end{quote}

\section{Bioprocessing}\label{SS.bioproc}

Before the advent of biotechnology, most therapeutics (medicines) were what are
now termed \emph{small molecule} drugs.
The pharmaceutical industry was concerned with the synthesis of these products
via predominantly chemical processes, such as reaction, distillation and 
crystallisation.
Small molecule drugs typically consist of tens or hundreds of atoms, such as
paracetamol, which has a molar mass of approximately 151 g/mol, or aspirin
(acetylsalicylic acid), which has a molar mass of approximately 180 g/mol.

With the advent of recombinant D.N.A technology, biochemists gained the ability
to re-program the D.N.A. of simple biological microorganisms such as
\textit{Escherichia coli} and, eventually, mammalian cells, such as those of
the Chinese Hamster (\textit{Cricetulus griseus}).
The genetic structure of these cells could be modified to produce complex
molecules, which had previously proved difficult or impossible to synthesise by
any other means.
The biopharmaceutical industry is concerned with the synthesis of therapeutics
via such biological pathways.
These products are known by various names, such as
\emph{biopharmaceuticals}, \emph{protein therapeutics} or, colloquially, as
\emph{biotech drugs}.

The first protein therapeutic to be synthesised on a large scale using 
biological pathways was insulin, which is a hormone used to regulate metabolism
and is administered to individuals suffering from diabetes.
Human insulin has a molar mass of approximately 5808 g/mol.
It was not possible to commercially synthesise insulin chemically and it was
initially produced by extracting the hormone from the pancreases of mammals
such as cows or pigs.
In 1978, scientists working at  the American company Genentech (now a
subsidiary of the Swiss pharmaceutical company F. Hoffmann-La Roche AG)
successfully modified cells of \textit{E. coli} to produce insulin and this
synthetic insulin was first brought to market in 1982.

The industry that has grown up around the production of biopharmaceuticals is 
known as the \emph{biopharmaceutical industry}, or, colloquially, as the
\emph{biotech} or \emph{biopharma} industry.
A report by strategy consultants McKinsey \& Company \citep{Otto:2014}
estimates that the biopharmaceutical market had global revenues of
\$163 billion per annum and was worth 20\% of the overall pharmaceutical
market.
\citet{Otto:2014} also note that large-scale biopharmaceutical manufacturing
facilities typically cost in the region of 
``\$200 million to \$500 million or more'' to build.

\section{Bioprocess Engineering}\label{SS.bioproceng}

\citet{Schaschke:2014} defines bioprocess engineering as ``A specialist branch
of (chemical) engineering that involves the design and operation of processes
used for the production of biological products such as foods, pharmaceuticals,
and biopolymers.''  
The design of a large-scale biopharmaceutical facility typically takes about
two years and requires a multidisciplinary team of engineers and scientists.

\section{Upstream and Downstream}\label{SS.updown}

A complete facility typically starts with frozen vials of cells (the 
\emph{working cell bank}) and finishes with the final formulated product in
either bulk form or filled into its final packaging e.g.\ syringes.
Facilities are nominally divided into \emph{upstream} and \emph{downstream} 
sections.
The upstream section is predominantly concerned with the expansion of cells
from a small vial into progressively larger tanks of \emph{media}.  
The final stage of this growth occurs in the \emph{production bioreactor}, 
wherein the conditions can be altered to encourage the cells to produce the 
target protein.

At the interface between upstream and downstream lies the \emph{harvest}
section.
In the harvest section, some initial separation is performed to begin
to isolate the target protein from the contents of the batch (which at this
point include cells, cell waste, growth media, antibiotics and myriad other
contaminants).
At the end of the harvest section, all traces of the host cells
should be removed and the batch is said to be \emph{cell-free}.
Different interpretations exist in the industry as to where the
upstream-downstream split occurs, but it usually is defined as being at some
point immediately before or after harvest.

Downstream processing is concerned with taking the cell-free but otherwise 
contaminated batch and purifying it through a series of orthogonal processes.
Such processes can include filtration, ultrafiltration/diafiltration 
(\emph{UF/DF}), chromatography, reaction, virus inactivation and formulation.
At the end of downstream processing, a batch should consist of formulated bulk
product, ready to be filled into its final packaging for delivery.
The final fill/finish steps often occur in a separate, sterile facility.

\section{Buffers and Media}\label{SS.buffmed}

Both upstream and downstream sections of a facility require the preparation of
large volumes of aqueous solutions.
Solutions used upstream are typically referred to as \emph{media} and those
used downstream are typically referred to as \emph{buffers}.
Strictly speaking, media refers to the solutions of nutrients into which cells
are expanded, but the term is usually used to encompass all other upstream
solutions, such as acids, bases and anti-foam used in the bioreactors.
Strictly speaking, a chemist would define a buffer as a solution which
maintains its pH over a wide range of concentrations.
Most solutions used downstream do indeed meet this criteria, but the term 
\emph{buffers} is generally used as a catch-all for all solutions used in the
downstream section of a facility.
A typical process to produce a \emph{monoclonal antibody} (a common family of 
protein therapeutics) can use tens or hundreds of litres of buffers and media 
per litre volume in the production bioreactor.
Typical large-scale production bioreactor volumes for such processes are in the
range \SIrange{10000}{30000}{\litre}.
Each batch may use in the region of \numrange[range-phrase=--]{20}{40} different
buffers and media.

\section{Buffers and Media Preparation}\label{SS.buffmedprep}
\begin{figure}
    \centering
    \includegraphics[angle=0,scale=1]{./figures/buffer_pfd.pdf}
    \caption{Single buffer preparation process flow diagram}
    \label{fig.pfd}
\end{figure}
One of the reasons for the catch-all definitions of \emph{buffers} and
\emph{media} in the section above is to do with segregation.
The upstream and downstream sections of the plant are segregated to prevent
cross-contamination.
As a result, there are typically two main areas where solutions are prepared.
Buffers are prepared in an area called \emph{buffer preparation}, for use
downstream and media are prepared in an area called \emph{media preparation},
for use upstream.
There may also be a separate area for preparing sterile buffers for the final
formulation, again to reduce the possibility of contamination and ensure
sterility is maintained.
In both media and buffer preparation, a preparation vessel is used to prepare
the solution and then it is typically sterile filtered into a separate hold
vessel.

\hyperref[fig.pfd]{Figure \ref*{fig.pfd}} is a \emph{process flow diagram}; a
simplified schematic that shows the main equipment used in buffer preparation.

Note that we define a complete processing step in a piece of equipment as a
\emph{procedure}, e.g.\ `preparation of buffer $n$'.
Each procedure consists of one or more \emph{operations}, e.g.\ `charge solids
to vessel'.
These definitions will be used throughout this document. 

A clean buffer preparation vessel is first charged with \emph{WFI} (water for
injection) -- an extremely pure quality water.
Solids are also charged to the vessel.
After closing the solids charging port, the contents of the vessel are
agitated to ensure complete solution of the solids.
Heating or cooling may be applied via the vessel jacket.
Sampling or testing may be performed, and adjustments made by adding more 
solids or WFI.

Once the buffer has passed all required quality checks, it is
pressure-transferred, via a sterile filter, to a sterile hold vessel.
Prior to this point, the sterile filter, the hold vessel and all
interconnecting pipework will have been sterilised, using clean steam, and then
dried.  
It is not possible to maintain sterility in the preparation vessel,
since open solids additions are required, so the sterile filter acts as a
barrier between the clean (but not sterile) preparation vessel and the
sterilised hold vessel.
Transfer of the buffer is usually performed by pressurising the preparation
vessel using sterile, filtered nitrogen (pumps are not desirable since they
contain complex interior surfaces that are hard to sterilise).
If the preparation vessel is capable of serving more than one hold vessel, the
buffer is directed via a \emph{valve ring} to its destination vessel.
A valve ring is simply a loop of pipe.
The valve ring has an open inlet port and several outlet ports.
Each outlet port has a specialised valve, called a \emph{zero dead-leg} valve,
which is located right at its junction with the loop.
The effect of this setup is that there are no points in the loop where liquid
can stagnate, which again aids in maintaining sterility.

Once the preparation vessel has finished its transfer of the buffer, it is
cleaned, ready for re-use.
The vessel is cleaned, along with some of its pipework, in an automated manner
known as \emph{CIP} (clean-in-place).
A piece of equipment known as a \emph{CIP skid} is used to perform the CIP.
A CIP Skid is connected to purified water and/or WFI supplies, as well as
to sources of cleaning chemicals, such as acids, bases and detergents.
The CIP skid contains a system of vessels, temperature control units and pumps
which are capable of sending these various solutions to destination equipment
in predefined cycles.  The waste generated during CIP is also returned to the
CIP skid via scavenging pumps, which direct waste to other waste handling
systems for further treatment or storage.

The sterilised hold vessel, having received the buffer, holds it until it is
required by the process.
If the buffer is made ahead of time, it may be held for several hours before
first use.
A buffer may be used by one or more operations in one or more procedures in the
production process.
Accordingly, it may be drawn discontinuously from the hold vessel during the
course of a batch.

Once the last use of the buffer has finished, the buffer hold vessel undergoes
a CIP.
The CIP leaves the vessel in a clean, but not sterile, state.
Just before the vessel is to be used again, it undergoes \emph{SIP}
(steam-in-place); clean steam is passed through the vessel and pipework until
it reaches a predefined temperature for a predefined hold time, after which the
vessel is allowed to cool down to room temperature.
Once the SIP is complete, the hold vessel is ready to receive the next lot of
buffer.

\section{Design of Buffer Preparation Areas}\label{SS.buffprepdes}

If a design for a large-scale production facility is to be carried out, the
client must provide some information on the nature of the production process.
Sometimes this can come in the form of information from another production
facility, where the product is already being produced. 
It may also come directly from small-scale development work carried out in a
laboratory, or a \emph{pilot plant} (a small plant used to demonstrate how a
process may be scaled from the laboratory to production scale).
Given this information, a process engineer can create a \emph{mass balance}, 
which is a document containing calculated values for all masses (and volumes)
of products, additions and wastes for each procedure in the process.
The mass balance is used to size the main production equipment (bioreactors,
purification equipment, etc.).
This mass balance provides, amongst other things, the volumes of all buffers
and media required to make a batch.
Two optimisation problems now emerge; how do we design the buffer and media
preparation areas?
For media, the problem is relatively easy to solve with some trial and error,
since there are typically only about ten media used per batch and they often
differ vastly in scale; the initial bioreactor may be \SI{20}{\litre} in volume
and the production bioreactor may be \SI{20,000}{\litre} in volume.
Because of this, the sizing of preparation vessels usually proceeds by picking
a vessel capable of preparing the largest medium, defining a minimum fill
volume and seeing what else can be prepared in it, then defining another
vessel, and so on until sufficient vessels are defined.
Media hold vessels, if required, may be similarly defined.
The design of media preparation is usually relatively insensitive to schedule.

The problem of designing a buffer preparation area is more difficult to solve.
There may be 20 or more different buffer compositions.
Often each buffer is used multiple times in the same procedure or multiple
times across multiple procedures.
In procedures such as chromatography, somewhere in the region of 
\numrange[range-phrase=--]{5}{10} buffers may be needed in rapid succession.
These chromatography buffers tend to be of similar volumes, so an efficient
solution will look to maximise the number of buffers prepared in a given
preparation vessel.
If several similarly-sized buffers are to be prepared in the same preparation
vessel, but all are required by the process concurrently, each will require a
dedicated hold vessel and several of them will have to be prepared ahead of
time, which puts pressure on the scheduling of their hold vessel activities.
There is thus a trade-off between the numbers and volumes of vessels and
the flexibility in scheduling buffers.

In practice, in the design of a buffer preparation area, there exists
trade-offs between efficiency and flexibility, capital and operating costs,
and many other factors such as installed area, installed volume, operability,
layout/adjacency, pipework complexity and cleanability.

Other factors that affect the design of the area are choices between fixed
equipment (typically in stainless steel) or disposable equipment (disposable
plastic preparation and hold bags are currently available in volumes up to
\SI{5000}{\litre}), and material compatibility issues (high-nickel alloys,
several times the cost of stainless steel, may be required for some buffers to
prevent corrosion).

\section{Problem Definition}\label{SS.probdef}
The task of designing a buffer preparation area is complex.
Current workflows are largely based on trial-and-error methods using a process
engineering scheduling software package (a selection of which are detailed in
\hyperref[SS.bioprocdes]{Section \ref*{SS.bioprocdes}}).
Using a chosen software package, a model is developed which captures the
scheduling of the main process procedures and also the scheduling of the
preparation and hold procedures for both buffer and media.
Typically, a conservatively large array of buffer vessels is chosen and the
schedule scenario is run.
If there is an individual vessel for each task, the scenario will resolve
easily, but the capital and space requirements of such a design will be
sub-optimal.
Via trial-and-error, individual preparation vessels may be removed or resized
and the model re-run.
The above process is iterated until it becomes difficult or impossible to
remove or reduce the preparation vessels any further and obtain a feasible
solution without scheduling clashes.
At this point, iteration stops.
It may appear that a `good' solution has been found at this point, but there
is no clear methodology to ensure that the solution is optimal.
% TODO: Pseudocode for this methodology 
In the early feasibility or concept stages of a project, this process is
cumbersome, the end points are poorly defined and any development of the
underlying process which varies the volumes required may necessitate starting
the optimisation again from scratch.
An additional factor is that a working solution may exist for a given
configuration, but the scheduling software is unable to resolve the problem,
giving a false negative. 
The scheduling tools used for the process tend to be deterministic, rather than
stochastic (although some sensitivity analysis is usually built in as an 
add-on); this makes it difficult to have confidence that a working schedule can
handle the real-world batch-to-batch variability inherent in a process that
depends on the productivity of living cells.

A more streamlined methodology for solving this optimisation problem is
required and this dissertation is concerned with developing such a methodology
and a software tool to implement it.
The aims are as follows:
\begin{itemize}
    \item Start with a reduced or basic case, including a number of simplifying
        assumptions
    \item Develop a tool to schedule the operations in buffer preparation and
        to vary the size and number of vessels and the preparation strategy to
        optimise the process with respect to some metric.
    \item Once a working framework has been developed, apply additional
        constraints (or remove simplifications) to provide a better
        approximation of real-world conditions.
\end{itemize}

In this study a technique is developed for design of the buffer preparation
process, including vessel selection, sizing, assignment and scheduling.
Unlike existing methods which utilise trial-and-error optimisation, the
proposed tool provably produces an optimum solution.
In addition, the algorithm converges to the optimum rapidly, for problems of
typical real-world complexity, providing a much faster design.
The proposed tool requires minimal information.
The information is requested in a format that is easy to understand, making it
accessible to process engineers without specific experience in using the
complex scheduling tools outlined in
\hyperref[SS.bioprocdes]{Section \ref*{SS.bioprocdes}}.

This study provides a novel approach for solving a vessel assignment problem,
and so contributes to the canon of knowledge in the areas of process
engineering and linear programming.

The business imperatives are twofold.
Firstly, for an engineering consultancy, the ability to rapidly, accurately and
repeatably solve such problems gives a competitive edge, which can be used to
win more business and to deliver more cost-effective design.
Secondly. for a biopharmaceutical client, optimising this problem results in
cost savings and having a well defined methodology for doing so gives
confidence that an in-progress design is indeed optimal and robust.

