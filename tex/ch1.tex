%   MSc Business Analytics Dissertation
%   Format based on skeleton template provided as part of module MIS40750
%
%   Title:     Optimising the design of buffer preparation in bioprocessing
%              facilities
%   Author:    Sean Tully
%
%   Chapter 1: Introduction
%
%   Change Control:
%   When     Who   Ver  What
%   -------  ----  ---  --------------------------------------------------------
%   06Jun16  ST    0.1  Begun 
%

\chapter{Introduction}\label{C.intro}

\begin{quote}
Well I don't think we're \emph{for} anything. We're just products of evolution.
You can say, `Gee, your life must be pretty bleak if you don't think there's a
purpose.' But I'm anticipating having a good lunch.

\hspace{2cm}--- James Watson, \emph{in conversation with Richard Dawkins}
\end{quote}

\section{Bioprocessing}\label{SS.bioproc}

Before the advent of biotechnology, most therapeutics (medicines) were what are
now termed \emph{small molecule} drugs. The pharmaceutical industry was
concerned with the synthesis of these products via predominantly chemical
processes, such as reaction, distillation and  crystallisation.  Small molecule
drugs typically consist of tens or hundreds of atoms, such as paracetamol, which
has a molar mass of approximately 151 g/mol, or aspirin (acetylsalicylic acid),
which has a molar mass of approximately 180 g/mol.

With the advent of recombinant D.N.A technology, biochemists gained
the ability to re-program the D.N.A. of simple biological microorganisms such as
\textit{Escherichia coli} and, eventually, mammalian cells, such as those of the
Chinese Hamster (\textit{Cricetulus griseus}).  The genetic structure of these
cells could be modified to produce complex molecules, which had previously
proved difficult or impossible to synthesise by any other means.  The
biopharmaceutical industry is concerned with the synthesis of therapeutics via
such biological pathways.  These products are known by various names, such as
\emph{biopharmaceuticals}, \emph{protein therapeutics} or, colloquially, as
\emph{biotech drugs}.

The first protein therapeutic to be synthesised on a large scale using 
biological pathways was insulin, which is a hormone used to regulate metabolism
and is administered to individuals suffering from diabetes.  Human insulin has a
molar mass of approximately 5808 g/mol.  It was not possible to commercially
synthesise insulin chemically and it was initially produced by extracting the
hormone from the pancreases of mammals such as cows or pigs.  In 1978, 
scientists working at  the American company Genentech (now a subsidiary of the
Swiss pharmaceutical company F. Hoffmann-La Roche AG) successfully modified 
cells of \textit{E. coli} to produce insulin and this synthetic insulin was
first brought to market in 1982.

The industry that has grown up around the production of biopharmaceuticals is 
known as the \emph{biopharmaceutical industry}, or, colloquially, as the
\emph{biotech} or \emph{biopharma} industry.  A report by strategy consultants
McKinsey \& Company \citep{Otto:2014} estimates that the biopharmaceutical 
market had global revenues of \$163 billion per annum and was worth 20\% of the
overall pharmaceutical market.  \citet{Otto:2014} also note that large-scale
biopharmaceutical manufacturing facilities typically cost in the region of
``\$200 million to \$500 million or more'' to build.

\section{Bioprocess Engineering}\label{SS.bioproceng}

\citet{Schaschke:2014} defines bioprocess engineering as ``A specialist branch
of (chemical) engineering that involves the design and operation of processes
used for the production of biological products such as foods, pharmaceuticals,
and biopolymers.''  The design of a large-scale biopharmaceutical facility
typically takes about two years and requires a multidisciplinary team of
engineers and scientists.

\section{Upstream and Downstream}\label{SS.updown}

A complete facility typically starts with frozen vials of cells (the 
\emph{working cell bank}) and finishes with the final formulated product in
either bulk form or filled into its final packaging \textit{e.g.} syringes.
Facilities are nominally divided into \emph{upstream} and \emph{downstream} 
sections.
The upstream section is predominantly concerned with the expansion of cells
from a small vial into progressively larger tanks of \emph{media}.  
The final stage of this growth occurs in the \emph{production bioreactor}, 
wherein the conditions can be altered to encourage the cells to produce the 
target protein.

At the interface between upstream and downstream lies the \emph{harvest}
section. In the harvest section, some initial separation is performed to begin
to isolate the target protein from the contents of the batch (which at this
point include cells, cell waste, growth media, antibiotics and myriad other
contaminants).  At the end of the harvest section, all traces of the host cells
should be removed and the batch is said to be \emph{cell-free}.  Different
interpretations exist in the industry as to where the upstream-downstream split
occurs, but it usually is defined as being at some point in the harvest section.

Downstream processing is concerned with taking the cell-free but otherwise 
contaminated batch and purifying it through a series of orthogonal processes.
Such processes can include filtration, ultrafiltration/diafiltration 
(\emph{UF/DF}), chromatography, reaction, virus inactivation and formulation.
At the end of downstream processing, a batch should consist of formulated bulk
product, ready to be filled into its final packaging for delivery. The final 
fill/finish steps often occur in a separate, sterile facility.

\section{Buffers and Media}\label{SS.buffmed}

Both upstream and downstream sections require large volumes of aqueous solutions
to be prepared and stored.  Solutions used upstream are typically referred to as
\emph{media} and those used downstream are typically referred to as
\emph{buffers}.  Strictly speaking, media refers to the solutions of nutrients
into which cells are expanded, but the term is usually used to encompass all
other upstream solutions, such as acids and bases anti-foam used in the
bioreactors.  Strictly speaking, a chemist would define a buffer as a solution
which maintains its pH over a wide range of concentrations.  Most solutions used
downstream do indeed meet this criteria, but the term \emph{buffers} is
generally used as a catch-all for all solutions used in the downstream section.
A typical process to produce a \emph{monoclonal antibody} (a common family of 
protein therapeutics) can use tens or hundreds of litres of buffers and media 
per litre volume in the production bioreactor.  Typical large-scale production
bioreactor volumes for such processes are in the range 10,000--30,000 litres.
Each batch may use in the region of 20--40 different buffers and media.

\section{Buffers and Media Preparation}\label{SS.buffmedprep}

One of the reasons for the catch-all definitions of \emph{buffers} and
\emph{media} in the section above is to do with segregation.
The upstream and downstream sections of the plant are segregated to prevent
cross-contamination.
As a result, there are typically two main areas where solutions are prepared.
Buffers are prepared in an area called \emph{buffer preparation}, for use
downstream and media are prepared in an area called \emph{media preparation},
for use upstream.
There may also be a separate area for preparing sterile buffers for the final
formulation, again to reduce the possibility of contamination and ensure
sterility is maintained.
In both media and buffer preparation, one vessel is used to prepare the solution
and then it is typically sterile filtered into either a hold vessel or the
destination vessel.

\section{Design of Buffer Preparation Areas}\label{SS.buffprepdes}

For a given product, a production process is defined at laboratory scale.  The
definition of key parameters at laboratory scale can allow process engineers to
generate a production-scale mass balance.  This mass balance provides, amongst
other things, lists of all buffers and media required to make a batch.  Two
complex optimisation problems now emerge; how do we design the buffer and media
preparation areas?  For media, the problem is relatively easy to solve with some
trial and error, since there are typically only about 10 media used per batch
and they often differ vastly in scale; the initial bioreactor may be 20 litres
in volume and the production bioreactor may be 20,000 litres in volume.  Because
of this, the sizing of preparation vessels usually proceeds by picking a vessel
capable of preparing the largest medium, defining a minimum fill volume and
seeing what else can be prepared in it, then defining another vessel, and so on
until sufficient vessels are defined.  Media hold vessels, if required, may be 
similarly defined.  The design of media preparation is usually relatively
insensitive to schedule.

The problem of designing a buffer preparation area is more difficult to solve.
There may be 20 or more different buffer compositions.  Often each buffer is
used multiple times in the same operation or multiple times across multiple
operations.  In operations such as chromatography, somewhere in the region of
5--10 buffers may be needed in rapid succession.  They tend to be of similar 
volumes, so an efficient solution will look to maximise the number of buffers
prepared in a given preparation vessel.  Since they may be needed in rapid
succession, this then involves a requirement for multiple hold vessels so
some buffers can be made ahead of time.  Where buffers are required for multiple
steps in an operation or across multiple operations, is it best to perform
many preparations, or few?  Additionally, is it best to hold the buffer
in many separate hold vessels, allowing them to be individually freed up more
quickly, or is it better to consolidate and minimised the number of vessels?
Defining success in the design of buffer preparation is also difficult.
There exist trade-offs between efficiency and flexibility, capital and 
operating costs, and many other factors such as installed area, installed 
volume, operability, layout/adjacency, pipework complexity and cleanability.

Due to the high salt concentrations in some buffers, they can prove corrosive
to the commonly used grades of stainless steel, such as 316L.  Such buffers
may have to be prepared or held in vessels made from expensive alloys such as
AL-6XN\textsuperscript{\textregistered}, which is about 3.5 times more
expensive than 316L, or a steel from the
Hastelloy\textsuperscript{\textregistered} family, which can be eight or more
times more expensive than 316L.
Ideally, the use of these alloys should be minimised.

Another factor is the use of disposable technology.  Buffer preparation, at
scales of up to 3,000 litres, can be carried out in disposable sterile bags,
rather than vessels.  These have a higher consumable cost but are faster to turn
around between preparations and reduce the utilisation of cleaning equipment.
Similarly, buffers can be held in disposable bags at volumes of up to 5,000
litre.  Development of disposables technology is currently rapid and the
available sizes and product ranges are increasing each year, so much so that 
the state of the art has often moved on between the finish of the detailed
design of a facility and the start of the first saleable production batch.

\section{Problem Definition}\label{SS.probdef}
\begin{figure}
    \centering
    \includegraphics[angle=0,scale=1]{./figures/buffer_pfd.pdf}
    \caption{Buffer preparation schematic}
    \label{fig.pfd}
\end{figure}


%TODO
\textbf{TODO: Rewrite this section}

The task of designing a buffer preparation area is complex.
Current workflows are largely based on trial-and-error methods using process
engineering scheduling software.  
Typically, a conservatively large array of vessels is chosen and the schedule
is run.  
If there is an individual vessel for each task, the schedule will resolve
easily, but the capital and space requirements will be onerous.
Via trial-and-error, individual vessels may be removed or resized and
the schedule re-run to see if it can be resolved.  
After some iteration, it becomes difficult or impossible to remove or reduce
the vessels any further and resolve the schedule.
At this point, iteration stops.
% TODO: Pseudocode for this methodology 
In the early feasibility or concept stages of a project, this process is
cumbersome, the end points are poorly defined and any development of the
underlying process which varies the volumes required may necessitate starting
the optimisation again from scratch.
An additional factor is that a working solution may exist for a given
configuration, but the scheduling software is unable to resolve the problem,
giving a false negative. 
The scheduling tools used for the process tend to be deterministic, rather than
stochastic (although some sensitivity analysis is usually built in as an 
add-on); this makes it difficult to have confidence that a working schedule can
handle the real-world batch-to-batch variability inherent in a process that has
living cells at its heart.S

A more streamlined methodology for solving this optimisation problem is
required and this dissertation is concerned with developing such a methodology
and a software tool to implement it.
The aim is to start with a reduced or basic case, including a number of
simplifying assumptions, and develop a working tool to schedule the operations
in buffer preparation and to vary the size and number of vessels and the 
preparation strategy to optimise the process with respect to some metric.
One a working framework has been developed, additional constraints can be added
and simplifications can be removed to provide a better approximation of 
real-world conditions.

Success will be defined both in terms of the ability for the software tool to
provide an optimum solution and the speed at which it can be implemented
relative to other methods or benchmarks.

The business imperatives are twofold.
For an engineering consultancy, the ability to rapidly, accurately and
repeatably solve such problems gives a competitive edge, which can be used to
win more business and to deliver designs more cheaply.  For a biopharmaceutical
client, optimising this problem results in cost savings and having a well
defined methodology for doing so gives confidence that an in-progress design is
indeed optimal and robust.

