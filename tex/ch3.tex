%   MSc Business Analytics Dissertation
%   Format based on skeleton template provided as part of module MIS40750
%
%   Title:     Optimising the design of buffer preparation in bioprocessing
%              facilities
%   Author:    Sean Tully
%
%   Chapter 3: Data
%
%   Change Control:
%   When     Who   Ver  What
%   -------  ----  ---  --------------------------------------------------------
%   06Jun16  ST    0.1  Begun 
%

\chapter{Data}\label{C.data}

\begin{quote}
We have some freedom in setting up our personal standards of beauty, but it is
especially nice when the things we regard as beautiful are also regarded by
other people as useful.

\hspace{2cm}--- Donald Knuth, \emph{Computer Programming as an Art}
\end{quote}

\section{Introduction}\label{S.intro3}

This section will outline what the input data might look like, including data
sources.
It will outline what form the output data and metrics should take.

For modeling a single process, a typical dataset consists of three files.
The first file is a table of data relating to the available selection of
vessels.
The second file is a table of data relating to parameters specific
to each buffer.
The third file comprises a collection of global parameters
that apply to all vessels and/or buffers.

\section{Vessel Data}\label{S.data1}

Typically, when designing a production facility, buffer preparation vessel 
volumes range from 1,000 l to 30,000 l.
It is usual to round the volume to the nearest 1,000 l.
When ordering such a vessel, the stated size of the vessel is usually a nominal
volume, which will differ from the liquid fill volume and may also differ from
the maximum working volume.
For the purposes of this study, it is assumed that the vessel volume is the
maximum permissible buffer preparation volume in a given vessel, e.g. a 
preparation vessel with vessel volume of 2,000 l cannot be used to prepare
buffers greater than 2,000 l.

Vessels will also have a minimum working volume. 
It is usal to assume a minimum fill ratio of about thirty percent of the vessel
volume. This limitation
arises due to the minimum agitation volume of the impeller in the vessel. 
For the purposes of this simulation, the minimum fill ratio is a global
parameter. It would be possible to specify a value of minimum fill ratio for
every vessel size, but this level of detail is typically neither required nor
available.

For each vessel, a (relative) cost must be defined.
Vessel cost does not scale linearly with volume.
For the purposes of this study, cost data for each vessel size has been
estimated by raising the vessel volume to a power of 0.6 and rounding to two
decimal places.
Note that absolute costs are not required to find the vessel selection that
minimises costs.

\section{Buffer Data}\label{S.data2}

At a minimum, the volume of each buffer is required.
If nothing was known about the scheduling of the prodcution process, a simple
simulation could be carried out with scheduliung unconstrained, save for a
maximum buffer preparation vessel utilisation factor.
This factor is a limit on the fraction of a cycle for which a given vessel is
utilised.  A 
