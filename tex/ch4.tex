%   MSc Business Analytics Dissertation
%   Format based on skeleton template provided as part of module MIS40750
%
%   Title:     Optimising the design of buffer preparation in bioprocessing
%              facilities
%   Author:    Sean Tully
%
%   Chapter 4: Methodology
%
%   Change Control:
%   When     Who   Ver  What
%   -------  ----  ---  --------------------------------------------------------
%   06Jun16  ST    0.1  Begun 
%

\chapter{Methodology}\label{C.methodology}

\begin{quote}
Just as the largest library, badly arranged, is not so useful as a very moderate
one that is well arranged, so the greatest amount of knowledge, if not
elaborated by our own thoughts, is worth much less than a far smaller volume
that has been abundantly and repeatedly though over.  For only by universally
combining what we know, by comparing every truth with every other, do we fully
assimilate our own knowledge and get it into our power.

\hspace{2cm}--- Arthur Schopenhauer, \emph{On Thinking for Oneself}
\end{quote}

\section{Introduction}\label{S.intro4}

The vessel selection problem may be described as a series of linear constraints.
These constraints are applied to find the optimium value of an objective
function, which we seek to \emph{minimise}.
The objective function is the total cost of vessels, which is given by:

\begin{equation}
    Z = \sum_{m \in M} \sum_{p \in P} c_m \boldsymbol{y}_{mp}
\end{equation}

A small number of constraints need to be applied to arrive at the simplest 
variant of the problem.
Additional constraints may then be added to make the model more detailed or
realistic.

The first constraint to be added is the limitation that a buffer must be 
prepared in exactly one slot. 
\textbf{TODO: Explanation of slots, schematic etc. before this point.}
This constraint means that we must indeed preprare each buffer once per cycle
and also that the buffer is always prepared in the same vessel -- i.e. vessel
selection for every cycle is identical.

\begin{equation}
    \sum_{p \in P} \boldsymbol{x}_{np} = 1 \quad \forall n \in N
\end{equation}

The second constraint to be added is the requirement that at most one vessel
may inhabit a given slot.
This is not directly analagous to the first constraint -- it is possible to 
use the same \emph{sized} vessel in many slots, but a maximum of one vessel 
\emph{instance} may inabit any given slot.
Note that this inequality allows for the possibility of unused slots, i.e. the
number of occupied slots (and hence the number of preparation vessels) may be
less than the number of available slots (and hence the number of buffers).

\begin{equation}
    \sum_{m \in M} \boldsymbol{y}_{mp} \le 1 \quad \forall p \in P
\end{equation}

The third constraint is the requirement that, if a vessel is in a given slot,
it has sufficient volume to prepare all buffers assigned to the slot.   
        
\begin{equation}
    \sum_{m \in M} U_{n} \boldsymbol{x}_{np} - V_{m} \boldsymbol{y}_{mp} \le 0
    \quad \forall n \in N, \quad \forall p \in P
\end{equation}

The fourth constraint required for a basic model is the limitation that the
total duration of each hold prcedure must not be greater than the cycle time.
If this constraint is not observed, a hold procedure in a given batch may not
have finished before the hold procedure for the next batch is due to start.

\begin{equation}
    \boldsymbol{z}_{n} \le \lambda - \left( \Delta t_{HOLD,PRE} +
    \Delta t_{TRANSFER} + \Delta t_{USE,n} + \Delta t_{HOLD,POST} \right)
    \quad \forall n \in N
\end{equation}


The final constraint required for a basic model is similar to the above.
We wish to ensure that, in a given slot, the sum of the total durations of each
preparation procedure in that slot is not greater than the cycle time.
If this were not the case, for a given slot, there would be insufficient
time to carry out all the preparations assigned to that slot.
Note that at this point, we have not concerned ourselves with \emph{when}
the buffers are required by the process -- indeed, perhaps this information is
not available early in a design project.
To make the model more realistic, we can modify this constraint so that sum of
the total durations mentioned above is not greater than some fraction of the
cycle time, the \emph{maximum utilisation ratio}.
By applying this constraint, we are saying that, in the absence of any detailed
scheduling data, we want to ensure that out preparation vessels
are used less than, e.g. 60\% of the time.

\begin{equation}
    \Delta t_{PREP,TOTAL} \sum_{n \in N} \boldsymbol{x}_{np} \le r_{UTIL} 
    \lambda \quad \forall p \in P
\end{equation}

where

\begin{equation}
    \Delta t_{PREP,TOTAL} = \Delta t_{PREP,PRE} + \Delta t_{TRANSFER} +
    \Delta t_{PREP,POST}  
\end{equation}

The above model may be solved to produce a vessel selection.
The resultant selection would be a rough guide, in the absence of production
scheduling information, of the number and size of vessels required for a 
facility, using the maximum utilisation ratio as a proxy for a schedule.

For a more accurate appraisal of vessel requirements, more data is required
on scheduling.
Specifically, data are required on the duration of use of each buffer, along
with data on the time of first use of each buffer, relative to some fixed point
in a batch (e.g. batch start).
Given this information, it is possible to constrain the problem so that the
individual preparations are scheduled correctly.
The scheduling constraint may be descibed quite simply:
Ensure that no two preparation operations overlap in time in a given slot.

Since we have a constant preparation duration, $ \Delta t_{PREP,TOTAL} $, this
constraint may be expressed, \emph{for any two distinct buffers that are made
in the same slot}, by the following:

\begin{equation}
    \lvert \left( t_{USE,k} - \boldsymbol{z}_{k} \right) - \left( t_{USE,n} - 
    \boldsymbol{z}_{n} \right) \rvert \le \Delta t_{USE,TOTAL} \quad \forall n 
    \in N, \quad \forall k \in N, k > n
\end{equation}

Note that the range of $ k $ is limited to $ k > n $ to prevent duplication
of constraints.
The above formula is not yet in a format that can be applied in a MILP.
Firstly, it was noted that the constraints only apply to two buffers which
happen to be made in the same slot.
Secondly, absolute value expressions are not valid in linear programming
constraints.
To overcome these issues, several additional constraints and variables must be
introduced.

Firstly, we want to specify a binary variable which indicates if two distinct
buffers are made in the same slot.  
This, in turn requres an additional binary variable which indicates if two
distinct buffers are made in a \emph{particular} slot.
The latter binary variable, $ \boldsymbol{w}_{nkp} $ is defined through a pair
of constraints:

\begin{equation}
    \begin{split}
        \begin{alignedat}{3}
            \boldsymbol{x}_{np} & {}+{} & \boldsymbol{x}_{kp} & {}-{} & 2 
            \boldsymbol{w}_{nkp} & \ge 0\\
            \boldsymbol{x}_{np} & {}+{} & \boldsymbol{x}_{kp} & {}-{} &
            \boldsymbol{w}_{nkp} & \le 1\\
        \end{alignedat}
    \end{split}
    \quad\quad
    \begin{split}
        \forall n \in N, \quad \forall k \in N, k > n
    \end{split}
\end{equation}

Given the above constraint, we can now define a variable, 
$ \boldsymbol{v}_{nk} $ which indicates if two distinct buffers are made in the
same slot.
This new varaible is introduced via the following constraint:

\begin{equation}
    \sum_{p \in P} \boldsymbol{w}_{nkp} \boldsymbol{v}_{nk} \le 0 \quad
    \forall n \in N, \quad \forall k \in N, k > n
\end{equation}

Recall that we still cannot apply our scheduling constraint due to the presence
of an absolute value expression in the equation.
The absolute value expression may be though of as representing a pair of 
constraints, e.g.
$ \lvert \alpha - \beta \rvert \ge \gamma $
is essentially shorthand for
$ \alpha - \beta \ge \gamma \quad \lor \quad \beta - \alpha \ge \gamma $. 
We now need to remove the logical-or ($\lor$) from the above pair of
inequalities.
This can be done by using the \emph{big-M} method, whereby a large constant,
$ \mathbb{M} $, is used to force selection of one or other of the constraints
based on the value of an additional binary.
In our case, the absolute value function represents two cases.
In one case, buffer $n$ is prepared before another buffer $k$, and in the
alternate case, buffer $k$ is prepared after buffer $n$.
We thus define a binary, $\boldsymbol{u}_{nk}$, such that
$ \boldsymbol{u}_{nk} = 0 $ iff $n$ is prepared before $k$ and
$ \boldsymbol{u}_{nk} = 1 $ iff $k$ is prepared before $n$.
In the edge case where the buffers are prepared at precisely the same time,
the binary may take either value.

We are not yet ready to define the constraint which governs the value of
$\boldsymbol{u}_{nk}$.
The reason for this is that it is difficult to define if one event happens
before or after another when they occur repeatedly in a cyclic process.
Recall that for each buffer, the scheduling data consist of a use start time
and a use duration.
From the point of view of the model, we are only concerned with the 
steady-state cyclic case, so we use a modified use start time:

\begin{equation}
    t_{USE,n}^{\prime} = t_{USE,n} \mod \lambda \quad \forall n \in N
\end{equation}

We want to now rigorously define whether an event, $\alpha$, occurs after
another event, $\beta$, iff $ \alpha \mod \lambda > \beta \mod \lambda $.
We are concerned with the timing of our preparation procedures.
Recall that all preparation procedures have the same durations.
Thus, when deciding which of two such procedures occurs first, we can use 
$ t_{USE,n}^{\prime} - \boldsymbol{z}_{n} $ to mark the timing of the
preparation procedure of a given buffer $n$.
Note that this may take a negative value, which must be corrected by adding a
factor of $\lambda$ to bring the vaue back into the single-cycle range, as we
can't use a modulo expression in an MILP constraint.
Another binary variable, $ \boldsymbol{q}_{n} $ is introduced to indicate if
$ t_{USE,n}^{\prime} - \boldsymbol{z}_{n} < 0 $, i.e. 

\begin{equation}
    \boldsymbol{q}_{n} =
    \begin{cases}
        1 \implies t_{USE,n}^{\prime} - \boldsymbol{z}_{n} \le 0\\
        0 \implies t_{USE,n}^{\prime} - \boldsymbol{z}_{n} \ge 0
    \end{cases}
    \quad \forall n \in N
\end{equation}

Note that the value of $ \boldsymbol{q}_{n} $ is undefined if 
$ t_{USE,n}^{\prime} - \boldsymbol{z}_{n} = 0 $.
The above definition of $ \boldsymbol{q}_{n} $ is captured in the following
pair of constraints:

\begin{equation}
    \begin{split}
        \begin{alignedat}{2}
            \boldsymbol{z}_{n} & {}-{} & \lambda \boldsymbol{q}_{n} & \le 
            t_{USE,n}^{\prime}\\
            \boldsymbol{z}_{n} & {}-{} & \lambda \boldsymbol{q}_{n} & \ge 
            t_{USE,n}^{\prime} - \lambda
        \end{alignedat}
    \end{split}
    \quad\quad
    \begin{split}
        \forall n \in N
    \end{split}
\end{equation}

Having incorporated $ \boldsymbol{q}_{n} $, we can now implement the pair of 
constraints which, given two distinct buffers, indicate which is prepared
first.

\begin{equation}
    \begin{split}
        \boldsymbol{z}_{n} - \boldsymbol{z}_{k} + \lambda \boldsymbol{u}_{nk}
        - \lambda \boldsymbol{q}_{n} + \lambda \boldsymbol{q}_{k} &\ge 
        t_{USE,n}^{\prime} - t_{USE,k}^{\prime}\\
        \boldsymbol{z}_{n} - \boldsymbol{z}_{k} + \lambda \boldsymbol{u}_{nk}
        - \lambda \boldsymbol{q}_{n} + \lambda \boldsymbol{q}_{k} &\le 
        t_{USE,n}^{\prime} - t_{USE,k}^{\prime} + \lambda
    \end{split}
    \quad \quad \forall n \in N, \quad \forall k \in N, k > n
\end{equation}

With the above constraint, we can finally implement the scheduling constraint:

\begin{equation}
    \begin{split}
        \begin{alignedat}{9}
        &\boldsymbol{z}_{n} {}-{} &\boldsymbol{z}_{k} {}+{} &\mathbb{M}
        \boldsymbol{u}_{nk} {}-{} &\mathbb{M} \boldsymbol{v}_{nk} &\ge
        &t_{USE,n}^{\prime} {}-{} &t_{USE,k}^{\prime} {}+{}
        &\Delta t_{PREP,TOTAL} {}-{} &\mathbb{M}\\
        - &\boldsymbol{z}_{n} {}+{} &\boldsymbol{z}_{k} {}-{} &\mathbb{M}
        \boldsymbol{u}_{nk} {}-{} &\mathbb{M} \boldsymbol{v}_{nk} &\ge
        - &t_{USE,n}^{\prime} {}+{} &t_{USE,k}^{\prime} {}+{}
        &\Delta t_{PREP,TOTAL} {}-{} 2 &\mathbb{M}
        \end{alignedat}
        \\\forall n \in N, \quad \forall k \in N, k > n
    \end{split}
\end{equation}

