%   MSc Business Analytics Dissertation
%   Format based on skeleton template provided as part of module MIS40750
%
%   Title:     Optimising the design of buffer preparation in bioprocessing
%              facilities
%   Author:    Sean Tully
%
%   Chapter 4: Methodology
%
%   Change Control:
%   When     Who   Ver  What
%   -------  ----  ---  --------------------------------------------------------
%   06Jun16  ST    0.1  Begun 
%

\chapter{Methodology}\label{C.methodology}

\begin{quote}
Just as the largest library, badly arranged, is not so useful as a very moderate
one that is well arranged, so the greatest amount of knowledge, if not
elaborated by our own thoughts, is worth much less than a far smaller volume
that has been abundantly and repeatedly though over.  For only by universally
combining what we know, by comparing every truth with every other, do we fully
assimilate our own knowledge and get it into our power.

\hspace{2cm}--- Arthur Schopenhauer, \emph{On Thinking for Oneself}
\end{quote}

\section{Introduction}\label{S.intro4}

In the context of the problem to be optimised, the term ``methodology'' could be
understood to cover either the methodology leading to the selection of a 
framework for solving the problem or to the framework itself.  This chapter
aims to cover both of these definitions and, in the process, outline the thought
processes underpinning the research.

\subsection{Methodology for developing a solution framework}\label{SS.method1}

The current workflow for designing buffer preparation areas involves choosing a
candidate set of vessels and apportioning particular operations to them in 
some scheduling package.  Using some engineering heuristics and a good deal of
trial and error, the selection and sizing of vessels and the strategy for
preparing buffers in them is iteratively improved upon.
It is proposed that this approach will be used as a starting point in developing
a methodology for solving the optimisation problem.

Thus the workflow would look something like this:
\begin{enumerate}
\item Solve schedule model for a candidate vessel matrix and buffer strategy
\item Observe results and identify clashes or vessels that could be removed
\item Resolve clashes or remove un-needed vessels and iterate
\end{enumerate}

The above method depends on iteratively performing a scheduling simulation over
with differing input parameters.  Such a framework would require two main
components, the first is a tool to schedule buffer operations, given some inputs
and the second is to use some heuristics to determine how the inputs should be
varied to optimise the design with respect to some metric.

Initially, these two elements should be designed to be as simple as possible, as
a proof of concept.  More detail will then be added once an initial working
framework is devised.  For instance, it is assumed that the main process
schedule is fixed.  This means that, relative to the start of a batch, it is
known when certain buffers are required by the process (both start and end
times) and the volume required in each instance is also known. 

A finite, small (less than 20) list of available vessel sizes is specified.
A minimum fill level (percentage) is defined that applies to all vessels.
A lit of available buffer preparation and hold vessels is somehow generated --
this may be a worst-case list, whereby each preparation has a dedicated vessel
and each hold has a dedicated vessel.  Preparation vessels cannot be used for
holding buffers and holding vessels cannot be used for preparing buffers.

It is initially assumed that all buffers are to be made once per operation, once
per batch.  For example, if we have a 20 batch campaign and a chromatography
operation requires \emph{Buffer A} for four different steps per cycle and the
chromatography colum processes a batch in three cycles, we only have to prepare
\emph{Buffer A} once per batch, \emph{i.e.} 20 times over the course of the 
campaign. A prepared buffer is stored in a single hold vessel - no buffer lots
are split or combined between preparation and hold.
In addition to information on the times when buffers are needed, the cycle time
of the main process is required (we assume that only one main process will be
running and that it has a fixed cycle time).  It is assumed that all operations
have a fixed duration and that they occur at fixed times relative to the start
of a batch \emph{i.e.} each batch is the same.  It is assumed that the
duration of individual steps in buffer preparation and hold vessels are 
independent of vessel volume or the volume of buffer being prepared.  
It is assumed that no step in buffer preparation or hold activities need compete
for an external resource, such as the availability of resources, such as labour
or utilties.  Buffers are assumed not to expire and allowable equipment clean 
and dirty hold times are assumed infinite (exceeding a clean equpment hold time
means the equipment is no longer considered sterile and must be re-cleaned,
exceeding a dirty equipment hold time means additional deep cleaning steps may
be required in addition to regular cleaning).  It is assumed that there are
no material compatibility issues, no interconnectivity constraints and no
buffers requiring additional segregation due to sterility or flammability
concerns.

The simplified model above will be input to some scheduling tool which will
aim to produce a working schedule for buffer preparation.  Note that one issue
here is that a working schedule for given inputs is not the only possible 
working schedule, which may detrimentally affect the ability to optimise
equipment - a strategy to cope with this fact must be developed.

A set of heuristics will be developed to examine the last known working schedule
and decide on what changes to make to the vessel matrix and buffer strategy.
The heuristic will need a metric it seeks to optimise. The simplest metric is
capital cost, which could be calculated based for each vessel based on its
volume and whether it is a preparation vessel (with impeller) or a hold vessel
(without impeller).  The heuristic may proceed, for example, by applying a
series of tests, allowing it to remove or resize vessels. One example would be 
if it can be seen that, \emph{e.g} there are four 2,000 litre hold vessels and
the operations that occur in each vessel do not overlap with the operations in
any of the other three, then three of the four vessels could be removed.
One heuristic might look at the largest size preparation required, which sets
the size of the largest vessel, then, looking at the idle times in that vessel
and looking at other preparations that are above the minimum fill level, it may
choose to move the other preparations to the largest vessel and removing some
other, smaller vessels.

Exhaustive enumeration of vessel sizes and counts (subject to some sensible
limits) should also be explored - if results can be found in reasonable time,
then this may be preferable as it would guard against finding a local optimum
and provide a greater degree of certainty in the results.

If a working framework can be developed, some of the simplifying assumptions can
be replaced with more detailed parameters and more complications can be added
to provide a tool that is better suited to simulating the detailed problems that
arise in practice in the industry.

\subsection{Methodology employed in the solution}\label{SS.method1}
This section will form the bulk of the research to be carried out and has not 
yet been commenced.
