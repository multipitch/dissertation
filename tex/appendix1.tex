%   MSc Business Analytics Dissertation
%   Format based on skeleton template provided as part of module MIS40750
%
%   Title:     Optimising the design of buffer preparation in bioprocessing
%              facilities
%   Author:    Sean Tully
%
%   Appendix 1: Running the Model
%
%   Change Control:
%   When     Who   Ver  What
%   -------  ----  ---  --------------------------------------------------------
%   06Jun16  ST    0.1  Begun 
%
\addappheadtotoc
\chapter{Program installation and usage}\label{C.Appendix1}
\begin{itemize}
\item
Python should be installed, along with the third-party python libraries
listed in \hyperref[tbl.libs]{Table \ref*{tbl.libs}}.
Note that the code in \texttt{model.py} requires version 3.3 (or newer) of
python.
The specific versions of all packages and libraries mentioned in
\hyperref[S.implementation]{Section \ref*{S.implementation}}
were found to be compatible.
It may be possible that breakages occur with other versions.
The pyenv package may be required to ensure the correct version of python is
available to the modelling code.
Optionally, additional
MILP solvers may be installed, such as those mentioned in
\hyperref[SS.impl1]{Section \ref*{SS.impl1}}.

\item
Clone the model repository by running the following from the terminal:
\texttt{\$ mkdir -p \textasciitilde/git\\
        \$ cd \textasciitilde/git\\
        \$ git clone \url{https://github.com/multipitch/dissertation}}

\item
To make the \texttt{runmodel} executable available to the user, it must be
appended to the user's \texttt{PATH} environment variable.
This can be achieved, for the active terminal session, by running the following
from within the terminal:

\texttt{\$ export PATH=\$PATH:<location>}

where \texttt{<location>} is the path to the directory containing
\texttt{runmodel} e.g.

\texttt{\$ export PATH=\$PATH:\textasciitilde/git/dissertation/src}

The above redefinition of \texttt{PATH} can be made permanent by appending the
command to e.g.\ the user's \texttt{\textasciitilde/.bashrc} file.

\item
To run using the default settings, given a directory containing valid input data
 (\texttt{buffers.csv},
\texttt{vessels.csv} and \texttt{parameters.ini} files), navigate to the
directory and run the following from the command line:

\texttt{\$ runmodel}

The above command uses PuLP's default solver (Cbc), without any arguments.
It solves for both the primary and secondary models and generates equipment
time utilisation plots for both, saving them to the directory as
\texttt{plot1.pdf} and \texttt{plot2.pdf}.

\item
Several optional parameters may be passed to the \texttt{runmodel} executable,
e.g.\ running

\texttt{\$ runmodel -s CPLEX -{}-no-secondary}

will run the model using the CPLEX solver and won't perform the secondary
optimisation.

A description of all input parameters is available by running \texttt{runmodel}
with the \texttt{-h} or \texttt{-{}-help} flags.

\item
Sample data is available in the examples directory of the repository.
The command below gives an example of how to carry out a run on one of the 
sample data sets:

\texttt{\$ runmodel -P ~/git/dissertation/examples/random/}
\end{itemize}
