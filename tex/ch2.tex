%   MSc Business Analytics Dissertation
%   Format based on skeleton template provided as part of module MIS40750
%
%   Title:     Optimising the design of buffer preparation in bioprocessing
%              facilities
%   Author:    Sean Tully
%
%   Chapter 2: Literature Review
%
%   Change Control:
%   When     Who   Ver  What
%   -------  ----  ---  --------------------------------------------------------
%   06Jun16  ST    0.1  Begun 
%

\chapter{Literature Review}\label{C.litreview}

\begin{quote}
It was long before I got at the maxim, that in reading an old mathematician you
will not read his riddle unless you plough with his heifer; you must see with
his light, if you want to know how much he saw.

\hspace{2cm}--- Augustus de Morgan, \emph{letter to W. R. Hamilton}
\end{quote}

\section{Introduction}\label{S.intro2}

In this chapter, recent papers related to the design of bioprocess facilities
are reviewed.  This is followed by the review of papers related to solving
similar problems in related industries.  Finally, papers relating to more
general methodologies for solving scheduling and optimisation problems are
reviewed.

\section{Bioprocess Facility Design}\label{SS.bioprocdes}

Current bioprocess facility design leverages software packages for creating mass
balances and for schedule simulation.  The mass balance tool
\emph{SuperPro Designer\textsuperscript{\textregistered}} 
and the scheduling tool \emph{SchedulePro\textsuperscript{\textregistered}}
from Intelligen, Inc. (\emph{Scotch Plains, New Jersey, U.S.A.}) are examples of
software packages that are widely used, particularly by American firms.
The \emph{INOSIM} family of software from INOSIM Software GmbH
\emph{Dortmund, Germany} is used particularly by German and Swiss firms.

\citet{Petrides:2014} outline a workflow for 
the design of a ``typical'' monoclonal antibody facility using 
\emph{SuperPro Designer\textsuperscript{\textregistered}} and
\emph{SchedulePro\textsuperscript{\textregistered}} and compares the use of
these packages with other methodologies.
While this paper does touch on buffer preparation,
it does not elaborate on a strategy for optimising the area, stating:
\begin{quote}In most real processes, buffer scheduling is considerably more
complex and challenging because of the larger number of buffers required for a
typical process (more than twenty), the shared use of pipe segments and transfer
panels, as well as constraints imposed by the limited availability of labor.
\end{quote}
\citet{Petrides:2014} cite an earlier paper by \citet{Toumi:2010} which also
looks at design of a facility for monoclonal antibody production.
\citet{Toumi:2010} also mention the difficulty of optimising the sizing and 
selection of buffer equipment, but do not outline a methodology for doing so.

\citet{Dietz:2008} outlines a genetic algorithm approach to optimising the 
design of protein production facilities, which is capable of dealing with
imprecise demands.  It is primarily looking at the specification of the main
process equipment and mass balance and it does not touch on buffer preparation.

No mentions could be found of methodologies for optimising buffer preparation in
bioprocess facility design.

\section{Facility Design Optimisation in the Process Industry}
\label{SS.fdopi}
Casting the net wider to the pharmaceutical, chemical and other process
industries yields several articles that deal with similar families of problems.

Of particular relevance to buffer preparation are efforts to simulate tank farm
design \citep{Al-Otaibi:2004, Stewart:2005, Sharda:2009, Terrazas-Moreno:2012}.
Tank farms commonly exist in large pharmaceutical, chemical and oil \&
gas facilities and consist of arrays of tens of tanks that are usually dedicated
to particular chemicals or products, but may be multi-use.  These tanks are used
to support the main process being carried out in the facility, similar to the
role of buffer hold vessels.  There are a great deal of papers that deal with
optimising throughput or productivity of existing tank farms or existing batch
processes, but very little articles relating to the optimisation of the design
of such processes.

\citet{Al-Otaibi:2004} describes efforts to optimise the design of a tank farm
for the oil \& gas industry using both linear programming and Monte Carlo
simulation methods.  Their brief magazine article does not provide any detail on
how the simulations were carried out.

\citet{Stewart:2005} cite the work of \citet{Al-Otaibi:2004} and outline a
method for optimising tank farm design using Monte Carlo simulation.  They
mention the use of a software tool called \emph{GRTMPS} from 
\emph{Haverley Systems, Inc., Houston, Texas, U.S.A}, supported by \emph{Excel}
spreadsheet software and \emph{Access} database software
(\emph{Microsoft Inc, Redmond, Washington, U.S.A}).
Again, this is a short magazine article.  It indicates that the scheduling 
produced useful results but does not give any technical detail as to how the
simulation was carried out.

\citet{Sharda:2009} outline the use of the discrete-event simulation tool
\emph{Arena\textsuperscript{\textregistered}} from
\emph{Rockwell Automation, Inc., Milwaukee, Wisconsin, U.S.A.} to optimise the 
utilisation of existing tank farm facilities and cite the work of
\citet{Sharda:2009}

\citet{Terrazas-Moreno:2012}, citing \citet{Stewart:2005} and 
\citet{Sharda:2009} provide a far more detailed description of efforts to
optimise tank farm operations using mixed integer linear programming 
\emph{MILP}.  Their work looks at optimising both schedule and tank selection, 
but is not strictly designed with the design of the facility itself, but rather
the optimal operation of a designed facility.

The work of \citet{Terrazas-Moreno:2012} provides some useful information on
techniques that could be applied to solve the problem of buffer preparation
area design, namely MILP and process scheduling.  Branching out further into
these fields yields more relevant literature.

\citet{Dedieu:2003} suggests a hybrid approach using genetic
algorithms and discrete-event simulation to address the problem of 
multi-objective batch plant design.

In \citet{Cavin:2004, Cavin:2005}, tabu search is discussed as a methodology to
optimise the design of multi-purpose batch plants.

\section{Simulation and Optimisation}\label{SS.simopt}

Casting the net wider still to look at techniques for solving generalised
scheduling and optimisation problems yields far more material, but further 
research is required to decide which techniques, if any, are relevant to the 
problem at hand.

There are two aspects to optimising the design.  The first aspect is scheduling,
as the ability to make a tank do more than one task will depend on the times
at which the tasks must (or may) occur.  The second aspect is selecting the
optimal sizes and numbers of vessels, which can be seen as a combinatorial
optimisation problem.

In his seminal 1957 paper, George Dantzig \nocite{Dantzig:1957} outlines several
types of combinatorial optimisation problems, one of which is the
\emph{knapsack problem}.  This problem relates to finding the most valuable
selection of objects that can be carried in a knapsack, subject to a total
weight limit, given a selection of candidate items, each having a weight and a 
value.
A whole range of knapsack-type problems have been researched in the intervening
period.  Detailed descriptions of the most common problems and the research 
carried out over the half century following Dantzig's paper are given by 
\citet{Korte:2012} and \citet{Martello:1990}.

One knapsack-type problem that may be of particular relevance is the bin-packing
problem.  \citet{Martello:1990} describes the bin packing problem as one in
which there are a number of items with associated weights and a number of bins
with associated capacities.  The aim is to assign each item to a bin so that the
weight capacity of the bin is not exceeded and the number of bins is minimised.
A number of algorithms for both exact and approximate solutions of the
bin-packing problem have been developed. \citet{Korte:2012} state that the
bin-packing problem is strongly \textbf{NP}-hard, indicating that efforts
should be taken to minimise the sample space when dealing with optimising vessel
selection.

\citet{Bettinelli:2010} investigates a particular variant of the bin packing
problem where there is a minimum filling constraint.  This is an important
consideration in vessel selection, usually some minimum fill level,
\textit{e.g.} 20--30\% is defined so that the impeller in the vessel remains
submerged during the mixing process and to reduce the volume of cleaning
solutions or water required to clean the vessel as a fraction of the volume of 
buffer produced.

In terms of process scheduling methodologies, a number of papers exist in the 
field of chemical engineering \citep{Ahmed:2000},

\citet{Ahmed:2000} states that the general process planning problem is also
\textbf{NP}-hard.

A detailed synopsis of scheduling methodologies applicable to the chemical 
and process industries is given by \citet{Harjunkoski:2014}.
