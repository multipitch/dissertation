%   MSc Business Analytics Dissertation
%   Format based on skeleton template provided as part of module MIS40750
%
%   Title:     Optimising the design of buffer preparation in bioprocessing
%              facilities
%   Author:    Sean Tully
%
%   Chapter 7: Conclusions
%
%   Change Control:
%   When     Who   Ver  What
%   -------  ----  ---  --------------------------------------------------------
%   06Jun16  ST    0.1  Begun 
%

\chapter{Conclusions}\label{C.conclusions}

\begin{quote}
I was not proud of what I had learned but I never doubted that it was worth
knowing.

\hspace{2cm}--- Hunter S. Thompson, \emph{The Rum Diary}
\end{quote}

The models derived in \hyperref[C.methodology]{Chapter \ref*{C.methodology}}
succeed in describing the buffer vessel selection problem.
The code which implements the models produces feasible
solutions from input data sets and gives both quantitative outputs (values for
all variables) and useful visual output (the equipment time utilisation plot).

Out of the MILP solvers investigated, it was found that CPLEX provided the
fastest and most robust solutions.

The number of equations in the complete model was found to be cubic with
respect to the number of buffers.  The number of variables in the complete
model was also found to be cubic in the number of buffers.
The time complexity of the complete problem was found to be exponential in
the number of vessels.

The basic model represents the simplest manifestation of the buffer vessel
selection problem, requiring a minimal amount of information
Additionally, the basic model has been built upon to arrive at the complete
model, which accounts for scheduling and accordingly requires some input data
on timings.

The secondary model builds upon the complete model and uses goal programming to
provide a more realistic schedule. Note that it does not improve on the primary
objective function value, i.e.\ the minimisation of vessel costs.

The models contain a number of simplifications.
Some of these simplifications are required due to the probable lack of
sufficient information.
One such example is fixed operation durations for operations such as
transfers.
If only a single model of pump were available, a flow rate could be
defined, so the transfer duration would scale linearly with the buffer volume.
In reality, given the range of volumes in a typical process, several sizes of
pumps and pipework may be specified so that pumps aren't oversized for small
transfers and pumps aren't so small that large transfers take an unacceptable
length of time.  As a result, the assumption that transfer duration is fixed
allows the selection of a conservative transfer time to cover all
eventualities.

Another assumption is that buffer volumes are fixed.
For example, if a buffer is used in two operations which occur in different
procedures at very different times in the cycle, we do not allow the
possibility that we may split this buffer into two separate preparations.
If, for instance, the particular buffer was the largest-volume buffer being
prepared by some margin, it may make sense to split it into two batches to
reduce the maximum size of preparation vessel required.
There are trade-offs here between the saving from reducing the volume of the
largest reparation vessel and the increased operating costs from performing an
extra preparation, in terms of personnel and the cost of cleaning utilities.
Apart from the difficulty for building such flexibility into the model, the
objective function would potentially become a good deal more difficult to
define, requiring knowledge on trade-offs between operating versus capital
cost. Such knowledge may not be readily available.

One additional feature which warrants further work is material compatibility.
Certain buffers may require the use of high-nickel alloys instead of the
usual grades of stainless steel; these alloys may be several times the price
of stainless steel.
A more resistant vessel can be used to prepare all buffers that may be prepared
in a less resistant vessel.
This would require a compatibility matrix between buffers and vessel materials
(the latter being a new dimension).
Additional logic would also be required to prevent incorrect vessel material
selection.

Other scope for further work includes the application of the model to
multi-product facilities.
Generally, a multi-product facility will run a campaign of batches of one
product, then switch over to run a campaign of some other process.
Such a facility would need buffer vessels capable of supporting each process.

Further exploration of results is possible through e.g. Monte Carlo simulation.
For instance, the sensitivity of a solution to variation in input data could
be explored, which would give a client greater certainty that a proposed
solution would be able to handle expected variations.

The methodology detailed in this report will be of use in future bioprocessing
facility design projects and adds to the body of knowledge in the fields of
process engineering and linear programming.
