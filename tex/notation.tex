%   MSc Business Analytics Dissertation
%   Format based on skeleton template provided as part of module MIS40750
%
%   Title:     Optimising the design of buffer preparation in bioprocessing
%              facilities
%   Author:    Sean Tully
%
%   List of Notation
%
%   Change Control:
%   When     Who   Ver  What
%   -------  ----  ---  --------------------------------------------------------
%   06Jun16  ST    0.1  Begun
%

\chapter*{List of Notation}\label{C.notation}

\vspace{0.5cm}

{\renewcommand{\arraystretch}{0.9}

\begin{longtabu} to \textwidth {X[-1,l] X[1,l]<{\strut} X[-1,r]}
    \tb{Symbol} & \tb{Description} & \tb{Ref}\\\hline
    \endhead
    $\boldsymbol{a}_{nk}$ & buffers $n$ and $k$ prepared in same
    vessel (binary) & \ref{SS.constr6}\\
    $c_{m}$ & (relative) cost of vessel $m$ & \ref{S.vesseldata}\\
    $f_{\mathit{MAXUSE}}$ & buffer maximum use time ratio 
        & \ref{SS.randomdata}\\
    $f_{\mathit{MINFILL}}$ & vessel minimum fill ratio & \ref{S.parameters}\\
    $f_{\mathit{MINUSE}}$ & buffer minimum use time ratio 
        & \ref{SS.randomdata}\\
    $f_{\mathit{UTIL}}$ & preparation slot maximum utilisation ratio
        & \ref{S.parameters}\\
    $k$ & secondary buffer index $\left( k \in \mathcal{N}, k \ne n \right)$
        & \ref{SS.schedintro}\\
    $m$ & vessel size index $\left( m \in \mathcal{M} \right)$ 
        & \ref{S.vesseldata}\\
    $n$ & buffer index $\left( n \in \mathcal{N} \right)$ 
        & \ref{S.bufferdata}\\
    $p$ & slot index $\left( p \in \mathcal{P} \right)$ & \ref{S.slots}\\
    $\boldsymbol{q}_{n}$ & the buffer $n$ hold operation crosses the
        single-cycle boundaries (binary) & \ref{SS.prepreftimes}\\
    $\boldsymbol{r}_{n}$ & $\boldsymbol{t}_{\mathit{LOWER},n}$ occurs before
        $\boldsymbol{t}_{\mathit{PREP},n}$ in the single-cycle window (binary)
        & \ref{SS.cyclic}\\
    $\boldsymbol{s}_{n}$ &
        $\boldsymbol{t}_{\mathit{UPPER},n}$ occurs after
        $\boldsymbol{t}_{\mathit{PREP},n}$ in the single-cycle window (binary)
        & \ref{SS.cyclic}\\
    $\boldsymbol{t}_{\mathit{LOWER},n}$ & lower bound of feasible scheduling
        region for all buffers $k > n$ with respect to buffer $n$
        & \ref{SS.cyclic}\\
    $\boldsymbol{t}_{\mathit{PREP},k}$ & preparation reference time for buffer
        $k$ & \ref{SS.prepreftimes}\\
    $\boldsymbol{t}_{\mathit{PREP},n}$ & preparation reference time for buffer
        $n$ & \ref{SS.prepreftimes}\\
    $\boldsymbol{t}_{\mathit{UPPER},n}$ & upper bound of feasible scheduling region for all
        buffers $k > n$ with respect to buffer $n$ & \ref{SS.cyclic}\\
    $t_{\mathit{USE},n}$ & buffer $n$ time of first use, normalised
        & \ref{S.bufferdata}\\
    $t_{\mathit{USE},n}^{*}$ & buffer $n$ time of first use, un-normalised 
        & \ref{S.bufferdata}\\
    $\Delta t_{\mathit{FEAS}}$ & maximum feasible buffer use duration
        & \ref{SS.randomdata}\\
    $\Delta t_{\mathit{HOLD,MAX}}$ & maximum allowable buffer hold duration
        & \ref{S.parameters}\\
    $\Delta t_{\mathit{HOLD,MIN}}$ & minimum allowable buffer hold duration
        & \ref{S.parameters}\\
    $\Delta t_{\mathit{HOLD,POST}}$ & duration of post-use operations in buffer
        hold procedures & \ref{S.parameters}\\
    $\Delta t_{\mathit{HOLD,PRE}}$ & duration of operations prior to receiving
        buffer in buffer hold procedures & \ref{S.parameters}\\
    $\Delta t_{\mathit{PREP}}$ & total duration of buffer preparation
        procedures & \ref{SS.constr4}\\
    $\Delta t_{\mathit{PREP,PRE}}$ & duration of operations prior to
        transferring out buffer in buffer preparation procedures & 
        \ref{S.parameters}\\
    $\Delta t_{\mathit{PREP,POST}}$ & duration of operations post transferring
        out buffer in buffer preparation procedures & \ref{S.parameters}\\
    $\Delta t_{\mathit{USE},n}$ & duration of use of buffer $n$ 
        & \ref{S.bufferdata}\\
    $\Delta t_{\mathit{TRANSFER}}$ & duration of transfers from buffer
        preparation vessel to buffer hold vessel & \ref{S.parameters}\\
    $\boldsymbol{u}_{n}$ & feasible scheduling window for 
        buffer $k$ with respect to buffer $n$  does not cross cycle
        boundary (binary) & \ref{SS.cyclic}\\
    $\boldsymbol{v}_{nk}$ & feasible scheduling window for buffer
        $k$ with respect to buffer $n$ occurs before buffer $n$ preparation
        procedure (binary) & \ref{SS.cyclic}\\
    $\boldsymbol{w}_{nkp}$ & distinct buffers $n$ and $k$ are both
        made in slot $p$ (binary) & \ref{SS.constr6}\\
    $\boldsymbol{x}_{np}$ & buffer $n$ is prepared in slot $p$ (binary)
        & \ref{SS.constr1}\\
    $\boldsymbol{y}_{mp}$ & a vessel of size $m$ is in slot $p$ (binary)
        & \ref{S.objfn}\\
    $\boldsymbol{z}_{n}$ & buffer $n$ hold duration & \ref{SS.constr5}\\
    $M$ & number of vessel sizes & \ref{S.vesseldata}\\
    $\mathcal{M}$ & set of vessel sizes & \ref{S.vesseldata}\\
    $N$ & number of buffers & \ref{S.bufferdata}\\
    $\mathcal{N}$ & set of buffers & \ref{S.bufferdata}\\
    $P$ & number of slots & \ref{S.slots}\\
    $\mathcal{P}$ & set of slots & \ref{S.slots}\\
    $T$ & process cycle time (start--to--start duration) & \ref{S.parameters}\\
    $U_{n}$ & volume of buffer $n$ to be prepared & \ref{S.bufferdata}\\
    $V_{m}$ & maximum working volume of vessel size $m$ & \ref{S.vesseldata}\\
    $V_{\mathit{MAX}}$ & largest maximum working volume of available vessel
        sizes & \ref{SS.constr3}\\
    $\boldsymbol{Y}$ & secondary objective; sum of buffer hold times,
        given minimal total vessel cost & \ref{SS.goal}\\
    $\boldsymbol{Z}$ & primary objective; total vessel cost & \ref{S.objfn}\\
    $Z^{\prime}$ & minimal total vessel cost & \ref{SS.goal}\\
\end{longtabu}

}

Notes:

Decision variables and objective function values are in 
$\boldsymbol{\mathit{bold}}$ text.

Subscripts in lower-case represent indices.
Subscripts in upper-case are descriptive.
Both subscript types may be used simultaneously, e.g. in the case of
$t_{USE,n}$.

A prime superscript denotes an optimum value, e.g. $Z^{\prime}$.

Lower-case Greek letters are used throughout the text as `throwaway' variables,
to explain certain concepts and are not included in the list of notation.
