%   MSc Business Analytics Dissertation
%   Format based on skeleton template provided as part of module MIS40750
%
%   Title:     Optimising the design of buffer preparation in bioprocessing
%              facilities
%   Author:    Sean Tully
%
%   List of Notation
%
%   Change Control:
%   When     Who   Ver  What
%   -------  ----  ---  --------------------------------------------------------
%   06Jun16  ST    0.1  Begun
%

\chapter*{List of Notation}\label{C.notation}

\vspace{0.5cm}

{\renewcommand{\arraystretch}{0.9}

\begin{longtabu} to \textwidth {X[-1,l] X[1,l]<{\strut} X[-1,r]}
    \tb{Symbol} & \tb{Description} & \tb{Ref}\\\hline
    \endhead
    $c_{m}$ & (Relative) cost of vessel $m$ & \ref{S.vesseldata}\\
    $k$ & Secondary buffer index $\left( k \in N, k \ne n \right)$
        & \ref{SS.constr6}\\
    $m$ & Vessel size index $\left( m \in M \right)$ & \ref{S.vesseldata}\\
    $n$ & Buffer index $\left( n \in N \right)$ & \ref{S.bufferdata}\\
    $p$ & Slot index $\left( p \in P \right)$ & \ref{S.slots}\\
    $\boldsymbol{q}_{n}$ & Buffer $n$ hold operation starts in cycle prior to
        the one in which it finishes & \ref{SS.constr8}\\
    $r_{\mathit{MINFILL}}$ & Vessel minimum fill ratio & \ref{S.parameters}\\
    $r_{\mathit{UTIL}}$ & Preparation slot maximum utilisation ratio
        & \ref{S.parameters}\\
    $t_{\mathit{USE},n}$ & Buffer $n$ time of first use, relative to batch
        start & \ref{S.bufferdata}\\
    $t_{\mathit{USE},n}^{*}$ & Buffer $n$ time of first use, relative to batch
        start, modulo cycle time & \ref{SS.constr8}\\
    $\Delta t_{\mathit{HOLD,MAX}}$ & Maximum allowable buffer hold duration
        & \ref{S.parameters}\\
    $\Delta t_{\mathit{HOLD,MIN}}$ & Minimum allowable buffer hold duration
        & \ref{S.parameters}\\
    $\Delta t_{\mathit{HOLD,POST}}$ & Duration of post-use operations in buffer
        hold procedures & \ref{S.parameters}\\
    $\Delta t_{\mathit{HOLD,PRE}}$ & Duration of operations prior to receiving
        buffer in buffer hold procedures & \ref{S.parameters}\\
    $\Delta t_{\mathit{PREP,PRE}}$ & Duration of operations prior to
        transferring out buffer in buffer preparation procedures & 
        \ref{S.parameters}\\
    $\Delta t_{\mathit{PREP,POST}}$ & Duration of operations post transferring
        out buffer in buffer preparation procedures & \ref{S.parameters}\\
    $\Delta t_{\mathit{PREP,TOTAL}}$ & Total duration of buffer preparation
        procedure & \ref{SS.constr4}\\
    $\Delta t_{\mathit{USE},n}$ & Duration of use of buffer $n$ 
        & \ref{S.bufferdata}\\
    $\Delta t_{\mathit{TRANSFER}}$ & Duration of transfer from buffer
        preparation vessel to buffer hold vessel & \ref{S.parameters}\\
    $\boldsymbol{u}_{nk}$ & Boolean: Buffer $k$ is prepared before buffer $n$
        & \ref{SS.constr8}\\
    $\boldsymbol{v}_{nk}$ & Boolean: Distinct buffers $n$ and $k$ are both made
        in the same slot& \ref{SS.constr7}\\
    $\boldsymbol{w}_{nkp}$ & Boolean: Distinct buffers $n$ and $k$ are both
        made in slot $p$ & \ref{SS.constr6}\\
    $\boldsymbol{x}_{np}$ & Boolean: Buffer $n$ is in slot $p$
        & \ref{SS.constr1}\\
    $\boldsymbol{y}_{mp}$ & Boolean: Vessel size $m$ is in slot $p$
        & \ref{S.objfn}\\
    $\boldsymbol{z}_{n}$ & Buffer $n$ hold duration & \ref{SS.constr5}\\
    $M$ & Number of vessel sizes & \ref{S.vesseldata}\\
    $N$ & Number of buffers & \ref{S.bufferdata}\\
    $P$ & Number of slots & \ref{S.slots}\\
    $U_{n}$ & Volume of buffer $n$ to be preapred & \ref{S.bufferdata}\\
    $V_{m}$ & Maximum working volume of vessel size $m$ & \ref{S.vesseldata}\\
    $V_{\mathit{MAX}}$ & Largest maximum working volume of available vessel
        sizes & \ref{SS.constr3}\\
    $\boldsymbol{Z}$ & Primary objective function value (minimal total vessel
        cost) & \ref{S.objfn}\\
    $\lambda$ & Process cycle time (start--to--start duration)
        & \ref{S.parameters}\\
\end{longtabu}

}

Notes:

Decision variables and objective function values are in $\boldsymbol{bold}$
text.

Subscripts in lower-case represent indices.
Subscripts in upper-case are descriptive.
Both subscript types may be used simultaneously, e.g. in the case of
$t_{USE,n}$.
