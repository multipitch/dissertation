%   MSc Business Analytics Dissertation
%   Format based on skeleton template provided as part of module MIS40750
%
%   Title:     Optimising the design of buffer preparation in bioprocessing
%              facilities
%   Author:    Sean Tully
%
%   Chapter 6: Discussion
%
%   Change Control:
%   When     Who   Ver  What
%   -------  ----  ---  --------------------------------------------------------
%   06Jun16  ST    0.1  Begun 
%

\chapter{Discussion}\label{C.discussion}

\begin{quote}
Nothing in life is as important as you think it is when you are thinking about
it.

\hspace{2cm}--- Daniel Kahneman,
\emph{Thinking, Fast and Slow}
\end{quote}

The models derived in \hyperref[C.methodology]{Chapter \ref*{C.methodology}}
succeed in describing the buffer vessel selection problem.
The code which implements the models produces feasible
solutions from input data sets and gives both quantitative outputs (values for
all variables) and a useful visual output (equipment time utilisation plot).

The basic model represents the simplest manifestation of the buffer vessel
selection problem, requiring a minimal amount of information
This has been built on by the complete model, which accounts for the scheduling
and accordingly requires some timing information.
The secondary model builds on the complete model to provide a more realistic
schedule, but does not improve on the primary objective, namely the
minimisation of vessel costs.

The model contains a number of simplifications.
Some of these simplifications are required due to the probable lack of
sufficient information.
One such example is fixed operation durations for operations such as
transfers.
If only a single model of pump were available, a flow rate could be
defined, so the transfer duration would scale linearly with the buffer volume.
In reality, given the range of volumes in a typical process, several sizes of
pumps and pipework may be specified so that pumps aren't oversized for small
transfers and pumps aren't so small that large transfers take an unacceptable
length of time.  As a result, the assumption that transfer duration is fixed
allows the selection of a conservative transfer time to cover all
eventualities.

Another assumption is that buffer volumes are fixed.
For example, if a buffer is used in two operations which occur in different
procedures at very different times in the cycle, we do not allow the
possibility that we may split this buffer into two separate preparations.
If, for instance, the particular buffer was the largest-volume buffer being
prepared by some margin, it may make sense to split it into two batches to
reduce the maximum size of preparation vessel required.
There are trade-offs here between the saving from reducing the volume of the
largest reparation vessel and the increased operating costs from performing an
extra preparation.
Apart from the difficulty for building such flexibility into the model, the
objective function would potentially become a good deal more difficult to
define, involving data on operating versus capital cost which may not be
available with any degree of accuracy.

One additional feature which warrants further work is material compatibility.
Certain buffers may require the use of high-nickel alloys instead of the
usual grades of stainless steel; these alloys may be several times the price
of stainless steel.
A more resistant vessel can be used to prepare all buffers that may be prepared
in a less resistant vessel.
This would require a compatibility matrix between buffers and vessel materials
(the latter being a new dimension).
Additional logic would also be required to prevent incorrect vessel material
selection.

Similar logic could be used in the case of flammable buffers.
Occasionally, ethanol is used in buffers. While the final buffer may not be
flammable, due to dilution, the handling of pure ethanol at the preparation
stage requires explosion-rated preparation vessels, which can be considerably
more expensive.
The vessels themselves may be similar, but all sensors, pumps, impellers,
actuated valves, etc.\ would need to be explosion-rated and this can add
significantly to the overall vessel cost.
If such a vessel were required, it could be used to produce non-flammable
buffers, but a non-explosion-rated vessel could not be used to handle flammable
materials.

Other scope for further work includes the application of the model to
multi-product facilities.
Generally, a multi-product facility will run a campaign of batches of one
product, then switch over to run a campaign of some other process.
Such a facility would need buffer vessels capable of supporting each process.
