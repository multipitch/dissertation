%   MSc Business Analytics Dissertation
%   Format based on skeleton template provided as part of module MIS40750
%
%   Title:     Optimising the design of buffer preparation in bioprocessing
%              facilities
%   Author:    Sean Tully
%
%   Preface
%
%   Change Control:
%   When     Who   Ver  What
%   -------  ----  ---  --------------------------------------------------------
%   06Jun16  ST    0.1  Begun 
%

\chapter*{Preface}

\begin{quote}
The cold smell of potato mould, the squelch and slap

Of soggy peat, the curt cuts of an edge

Through living roots awaken in my head.

But I've no spade to follow men like them.

Between my finger and my thumb

The squat pen rests.

I'll dig with it.

\hspace{2cm}--- Seamus Heaney, \emph{Digging}
\end{quote}

The motivation for this thesis was a desire to have a robust and efficient
method for optimising a problem encountered when attempting early-stage
design of bioprocessing facilities. 

% Introduction
\hyperref[C.intro]{Chapter \ref*{C.intro}} begins with an overview of the
field of bioprocessing, followed by an outline of the design challenges that
exist and an outline of the problem we are trying to solve; the buffer vessel
selection problem.

% Literature Review
In \hyperref[C.litreview]{Chapter \ref*{C.litreview}}, the relevant literature
is reviewed, starting with topics pertaining to bioprocess facility design and
branching into the broader topics of process optimisation and linear
programming.

% Data
\hyperref[C.data]{Chapter \ref*{C.data}} outlines the form of the input data
and describes the input parameters to the mathematical model.

% Methodology
In \hyperref[C.methodology]{Chapter \ref*{C.methodology}}, several
mixed-integer models are developed and the code used to implement them is
discussed.

% Results --- experimental, survey, etc
\hyperref[C.results]{Chapter \ref*{C.results}} catalogues the results obtained.
This includes outlining the form of the results, visualising the results and
investigation into the time taken to obtain a solution.

% Discussion
\hyperref[C.discussion]{Chapter \ref*{C.discussion}} discusses the results, 
and the complexity of the problem.

% Conclusions and future work
\hyperref[C.conclusions]{Chapter \ref*{C.conclusions}} outlines the conclusions
drawn from the exercise and the scope for further work.

\vspace{2em}

University College Dublin \hfill Sean Tully \\
\today 
