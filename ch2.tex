%   MSc Business Analytics Dissertation
%   Format based on skeleton template provided as part of module MIS40750
%
%   Title:     Optimising the design of buffer preparation in bioprocessing
%              facilities
%   Author:    Sean Tully
%
%   Chapter 2: Literature Review
%
%   Change Control:
%   When     Who   Ver  What
%   -------  ----  ---  --------------------------------------------------------
%   06Jun16  ST    0.1  Begun 
%

\chapter{Literature Review}\label{C.litreview}

\begin{quote}
It was long before I got at the maxim, that in reading an old mathematician you
will not read his riddle unless you plough with his heifer; you must see with
his light, if you want to know how much he saw.

\hspace{2cm}--- Augustus de Morgan, \emph{letter to W. R. Hamilton}
\end{quote}

\section{Introduction}\label{S.intro2}

In this chapter, recent papers related to the design of bioprocess facilities
are reviewed.  This is followed by the review of papers related to solving
similar problems in other industries.  Finally, papers relating to more
general methodologies for solving scheduling and optimisation problems are
reviewed.

\subsection{Bioprocess Facility Design}\label{SS.bioprocdes}

Current bioprocess facility design leverages software packages for creating mass
balances and for schedule simulation.  The mass balance tool
\emph{SuperPro Designer\textsuperscript{\textregistered}} 
and the scheduling tool \emph{SchedulePro\textsuperscript{\textregistered}}
from Intelligen, Inc. (\emph{Scotch Plains, New Jersey, U.S.A.}) are examples of
widely used software packages in the industry.
\citet{Petrides:2014} outlines a workflow for the design of a ``typical''
monoclonal antibody facility using 
\emph{SuperPro Designer\textsuperscript{\textregistered}} and
\emph{SchedulePro\textsuperscript{\textregistered}} and compares the use of
these packages with other methodologies (it should be noted that Petrides is the 
C.E.O. of Intelligen, Inc.).  While this paper does touch on buffer preparation,
it does not elaborate on a strategy for optimising the area, stating:
\begin{quote}In most real processes, buffer scheduling is considerably more
complex and challenging because of the larger number of buffers required for a
typical process (more than twenty), the shared use of pipe segments and transfer
panels, as well as constraints imposed by the limited availability of labor.
\end{quote}
\citet{Petrides:2014} cite an earlier paper by \citet{Toumi:2010} which also
looks at design of a facility for monoclonal antibody production.
\citet{Toumi:2010} also mention the difficulty of optimising the sizing and 
selection of buffer equipment, but do not outline a methodology for doing so.


\citet{Dietz:2008} outlines a genetic algorithm approach to optimising the 
design of protein production facilities, which is capable of dealing with
imprecise demands.  It is primarily looking at the specification of the main
process equipment and mass balance and it does not touch on buffer preparation.

No mentions could be found of methodologies for optimising buffer preparation in
bioprocess facility design.

\subsection{Facility Design Optimisation in the Process Industry}
\label{SS.fdopi}
Casting the net wider to the pharmaceutical, chemical and other process
industries yields several articles that deal with similar families of problems.


