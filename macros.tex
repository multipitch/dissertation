%   Macros
%   Esp. Theorem styles for amsart/amsbook document classes (currently enabled)
%
%   Sean McGarraghy
%
%   Change Control:
%   Date      Ver  Reason
%   --------  ---  --------------------------------------------------------------
%   12/05/98  0.1  Begun
%

\newcommand*{\Ax}  {Axiom~}
\newcommand*{\Df}  {Definition~}
\newcommand*{\Th}  {Theorem~}
\newcommand*{\Co}  {Corollary~}
\renewcommand*{\Pr}{Proposition~}
\newcommand*{\Lm}  {Lemma~}
\newcommand*{\Rk}  {Remark~}
\newcommand*{\Ex}  {Example~}
\newcommand*{\Exe} {Exercise~}
\newcommand*{\Eq}  {Equation~}
\newcommand*{\dfn} {definition}

\theoremstyle{plain}                 % Default

\newtheorem{thm}{Theorem}[chapter]
\newtheorem{lemma}[thm]{Lemma}
\newtheorem{lem}[thm]{Lemma}
\newtheorem{propn}[thm]{Proposition}
\newtheorem{pr}[thm]{Proposition}
\newtheorem{prp}[thm]{Propn.}
\newtheorem{cor}[thm]{Corollary}
\newtheorem{fact}[thm]{Fact}
\newtheorem{conj}[thm]{Conjecture}
\newtheorem{claim}{Claim}[]

\theoremstyle{definition}

\newtheorem{dm}{}  % Dummy theorem name for numbering only, independent of section
\newtheorem{dmy}[thm]{}  % Dummy theorem name for numbering only
\newtheorem{defn}[thm]{Definition}
\newtheorem{ex}[thm]{Example}
\newtheorem{exe}[thm]{Exercise}

\theoremstyle{remark}

\newtheorem{rmk}[thm]{Remark}
\newtheorem{notn}[thm]{Notation}
\newtheorem{note}[thm]{Note}
\newtheorem{cmt}[thm]{Comment}
\newtheorem{case}[thm]{Case}
\newtheorem{intr1}[thm]{Introduction}
\newtheorem{intro}[thm]{Introduction}

\newenvironment{pf}{\vspace{-4pt}\noindent\begin{proof}}{\end{proof}\vspace{2pt}}



